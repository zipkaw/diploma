% Contains config of surnames & other
% "~" - неразрывный пробел
% "\ " - Явно поставить пробел после команды
% Можно переиспользовать созданные выше переменные

% Даты, приказы и номера
\newcommand{\targetYear}{2024}
%% Приказ
\newcommand{\uniDecreeDate}{25 марта \targetYear\ г.}
\newcommand{\uniDecreeNumber}{642-с}
\newcommand{\taskStartDate}{25 марта \targetYear\ г.}
\newcommand{\taskFinishDate}{1 июня \targetYear\ г.}
\newcommand{\diplomaVariant}{083}
\newcommand{\econCalcDate}{14.04.\targetYear\ г.}

% Люди
\newcommand{\studentShort}{Э.И.~Сенкевич}
\newcommand{\studentFullParental}{Сенкевича Эдварда Ивановича}
\newcommand{\studentFullTask}{Сенкевичу Эдварду Ивановичу}
\newcommand{\stdTestTutorShort}{Н.О.~Туровец}
\newcommand{\headOfDepartmentShort}{Б.В.~Никульшин}
\newcommand{\practiceTutorShort}{А.В.~Бушкевич}
\newcommand{\practiceDepartmentTutorShort}{И.И.~Глецевич}
\newcommand{\diplomaTutorShort}{\practiceTutorShort}
\newcommand{\diplomaDepartmentTutorShort}{\practiceDepartmentTutorShort}
\newcommand{\diplomaEconomyTutorShort}{В.Г.~Горовой}

% Названия
\newcommand{\taskNameFull}{Интеграция и обработка данных инерциальных датчиков для определения позиции и азимута}
\newcommand{\economicalPartName}{Технико-экономическое обоснование интеграции и разработки модуля
    обработки данных инерциальных датчиков для определения позиции и азимута}
\newcommand{\rub}{р.}
\newcommand{\iec}{IEC}
\newcommand{\isoIec}{ISO/IEC}
\newcommand{\iso}{ISO}
\newcommand{\ieee}{IEEE}
\newcommand{\iecStd}{\iec\ 61850}
\newcommand{\libIec}{libiec61850}
\newcommand{\libXml}{libxml2}
% Также ссылаемся на нужную часть библиографии
\newcommand{\iecStdRef}[2]{\iecStd-{#1}-{#2}~\cite{IEC61850_#1_#2}}
\newcommand{\xml}{XML}
\newcommand{\cid}{ICD}
\newcommand{\xsd}{XSD}

% Схемы
\newcommand{\structScheme}{ГУИР.400201.\diplomaVariant \ С1}
\newcommand{\dataScheme}{ГУИР.400201.\diplomaVariant \ ПД.1}
\newcommand{\blockScheme}{ГУИР.400201.\diplomaVariant \ ПД.2}
\newcommand{\seqIcdScheme}{ГУИР.400201.\diplomaVariant \ РР.1}
\newcommand{\seqGooseScheme}{ГУИР.400201.\diplomaVariant \ РР.2}

% Модули структурного проектирования
\newcommand{\modulePerifery}{взаимодействия с периферией}
\newcommand{\moduleCalib}{калибровки устройства}
\newcommand{\moduleCalibControl}{управления калибровкой}
\newcommand{\moduleUart}{приема команд по UART}
\newcommand{\moduleMoveDetect}{распознавания движения}
\newcommand{\moduleOrientationAzimuth}{расчета ориентации и азимута}
\newcommand{\moduleFindTarget}{поиска цели на экране}
\newcommand{\moduleFlashMemory}{взаимодействия с флэш памятью}
\newcommand{\moduleGraphics}{графики}

% Экономика
\FPeval{\configRoundSigns}{2}
\FPeval{\configPercentRoundSigns}{0}

% Остальное
\newcommand{\rubFormula}{\text{ \rub}}
