\sectionCenteredToc{ЗАКЛЮЧЕНИЕ}
\label{sec:outro}

В рамках данного дипломного проекта разработан и протестирован модуль обработки инерциальных датчиков для определения позиции и азимута, а также была произведена интеграция
разрабатываемого программного обеспечения в разрабатываемый проект компании ООО <<АКСОНИМ>>. 
Также в ходе дипломного проектирования были добавлены дополнительные возможности улучшающие как работу программного обеспечения, так и возможность взаимодействия с пользователем.
Разработанная система имеет следующие преимущества:

\begin{itemize}
    \item модуль реализует качественную калибровку всех датчиков, что позволяет использовать более дешевые датчики для получения наилучшего результата.
    \item модуль использует асинхронное взаимодействие программных компонентов без использования средств операционной системы.
\end{itemize}

К недостаткам и ограничениям дипломного проекта относятся:

\begin{itemize}
    \item отсутствие проведения тестирования всех компонентов сразу, а не по отдельности;
    \item калибровка акселерометра требует от пользователя внимательности из-за большого количества шагов.
\end{itemize}

Первый недостаток связан с тем, что система должна быть протестирована целиком, с целью выявления ошибок в работе модулей между друг другом.
Второй пункт является особенностью алгоритма калибровки акселерометра, где требуется большое количество шагов для правильной калибровки,
ошибка со стороны пользователя может привести к неверным показаниям азимута и ориентации.

С учетом входных ограничений дипломный проект разработан в
полном объеме и качественно выполняет возложенные на него функции. В
дальнейшем он может быть улучшен следующими способами:

\begin{itemize}
    \item оптимизация способа калибровки акселерометра;
    \item добавление интеграционных тестов;
    \item добавление эмуляции работы устройства с целью проверки работы устройства независимо от аппаратной части;
    \item добавление версий продукта, который рассчитан на другие операционные системы и проверен на них при помощи всевозможных тестов.
\end{itemize}

Таким образом, разработанный модуль обработки инерциальных датчиков для определения позиции и азимута был интегрирован в продукт разрабатываемый компанией 
ООО <<АКСОНИМ>>. А также имеет перспективы на успешное использование в других коммерческих продуктах.
