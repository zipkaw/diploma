\referenceTitle

Дипломный проект предоставлен следующим образом. Электронные носители: 1
компакт-диск. Чертежный материал: 6 листов формата А1. Пояснительная записка:
\insertNumPagesText ,
\insertNumFiguresText , \insertNumTablesText ,
\insertNumBibElementsText ,
\insertNumAnnexesText .

Ключевые слова: калибровка, инерциальные датчики, обработка, фильтрация, азимут, ориентация.

Предметной областью данной разработки являются наблюдательные приборы, беспилотные средства.
Объектом разработки является калибровка датчиков для определения позиции и азимута.

Целью данного дипломного проекта является разработать платформонезависимый модуль обработки данных инерциальных датчиков, включающий, калибровку, фильтрацию, определение позиции и азимута, отображение.

Для разработки использовались язык программирования C,
система сборки CMake.
В качестве среды разработки
был взят редактор исходного кода Visual Studio Code.
Для тестирования применялись библиотеки Unity и CMock.
В качестве инструментов для компиляции исходных файлов для целевой платформы использовались инструменты Vivado и Vitis.

Разработанный модуль производит прием данных, их калибровку, фильтрацию,
производит вычисление азимута и ориентации прибора. Выполняет определение
движения и отображение данных на экране. Также разработанный модуль имеет
возможность работать с флэш памятью и UART, что позволяет загружать прошивку, читать
и записывать данные в память.
Система позволяет пользователю производить калибровку когда ему это необходимо.

Разработка данного программного продукта является эффективной
и позволяет уменьшить затраты как компании-заказчика, так и компании-разработчика в перспективе.

Проект завершен и полностью готов к внедрению в коммерческие проекты.
Система имеет несколько направлений развития, которые могут быть реализованы
по требованию за приемлемые сроки.

\newpage
