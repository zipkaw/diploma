\section{РАЗРАБОТКА ПРОГРАММНЫХ МОДУЛЕЙ}
\label{sec:dev}

В разделе разработки программных модулей
описана разработка ключевых алгоритмов для данного дипломного проекта.

\subsection{Алгоритм калибровки магнетометра}

Описываемый алгоритм отвечает за калибровку магнетометра и описывает полный цикл работы модуля калибровки магнитометра: от инициализации и вычитки параметров до
вычисления калибровки и уточнения данных. 

Распишем данный алгоритм по шагам:
\begin{enumerate_step}
    \item Начало алгоритма.
    \item Проверка дескриптора, что он не равен невалидному адресу \lstinline|BL_CALIB_INVALID_HANDLE|, если равен, то выйти с ошибкой
    \lstinline|BL_CALIB_ERROR_INVALID_HANDLE|.
    \item Создание массива \lstinline|calibrateData| для хранения откалиброванных данных.
    \item Создание массива \lstinline|hardIronCalibrate| для хранения калибровочных параметров.
    \item Создание массива \lstinline|hardIronCalibrate| для хранения жестких смещений магнитного поля.
    \item Запуск потока вызовом \lstinline|PT_BEGIN|.
    \item \label{i:alg:mag:965} Проверка существует ли запрос на чтение параметров из флэш памяти. Если поле \lstinline|readParametersRequest| дескриптора равно \lstinline|CALIB_REQUEST_FOR_PARAMETERS|,
    то перейти к шагу \ref{i:alg:mag:967}, иначе перейти к следующему шагу.
    \item \label{i:alg:mag:974} Если текущее поле \lstinline|command| дескриптора равно \lstinline|CALIB_APPLY|, то перейти дальше, иначе перейти к шагу
    \ref{i:alg:mag:997}.
    \item Проверить поле структуры обратного вызова дескриптора \lstinline|magnetoCb.magnetoWasRead|, на то были ли получены новые данные от драйвера магнетометра, 
    если нет то переключить контекст на другие потоки, иначе выполнить следующий шаг.
    \item Снять флаг \lstinline|magnetoCb.magnetoWasRead|.
    \item Проверить была ли ошибка при получении данных от драйвера магнитометра. Если ошибка была то перейти к следующему шагу, иначе перейти к шагу \ref{i:alg:mag:986}.
    \item В случае ошибки вызвать обратный вызов с кодом ошибки \lstinline|BL_CALIB_MANAGER_ERROR_MAGNETOMETER| и данными равными \lstinline|NULL|,
    который хранится в поле обратного вызова \lstinline|cbParams.cbWithData| дескриптора.
    \item \label{i:alg:mag:986} Вычесть вектор смещений \lstinline|hardIronOffsets| по трем осям из прочитанного вектора \lstinline|data|.
    \item \label{i:alg:mag:991} Перемножить матрицу \lstinline|softIronScales|, хранящую мягкие смещения магнетометра вместе с вектором полученном на шаге \ref{i:alg:mag:986}
    \item Вызвать пользовательскую функцию \lstinline|cbParams.cbWithData| со статусом \lstinline|BL_CALIB_MANAGER_OK| и данными полученным на шаге \ref{i:alg:mag:991}
    \item \label{i:alg:mag:997} Если текущее поле \lstinline|command| дескриптора равно \lstinline|CALIB_RECALIBRATE_CANCEL|, то перейти дальше, иначе перейти к шагу
    \ref{i:alg:mag:1004}.
    \item Установить в поле \lstinline|command| дескриптора команду \lstinline|CALIB_APPLY|.
    \item Установить в поле \lstinline|measurementCount| дескриптора значение 0.
    \item Вызвать функцию пользователя \lstinline|cbWithData| со статусом \lstinline|BL_CALIB_MANAGER_OK|.
    \item \label{i:alg:mag:1004} Если текущее поле \lstinline|command| дескриптора равно \lstinline|CALIB_RECALIBRATE|, то перейти дальше, иначе перейти к шагу
    \ref{i:alg:mag:1008}.
    \item \label{i:alg:mag:810} Проверить, что поле \lstinline|command| дескриптора не равно \lstinline|CALIB_RECALIBRATE_CANCEL|, если условие истинно, то перейти к следующему шагу, иначе
    перейти к шагу \ref{i:alg:mag:965}.
    \item Проверить, что \lstinline|measurementCount| меньше \lstinline|CALIB_MEAS_NUM|, если условие истинно, то перейти дальше, иначе перейти к шагу \ref{i:alg:mag:830}.
    \item Проверить поле структуры обратного вызова дескриптора \lstinline|magnetoCb.magnetoWasRead|, на то были ли получены новые данные от драйвера магнетометра, 
    если нет, то переключить контекст на другие потоки, иначе выполнить следующий шаг.
    \item Снять флаг \lstinline|magnetoCb.magnetoWasRead|.
    \item Сохранить полученные данные в буфер \lstinline|calib_dataStorage|.
    \item Увеличить счетчик measurementCount на единицу.
    \item Перейти к шагу \ref{i:alg:mag:810}.
    \item \label{i:alg:mag:830} Проинициализировать значения \lstinline|magnetColl| для этого необходимо для каждого вектора из буфера \lstinline|calib_dataStorage| найти его сонаправленный.
    \item \label{i:alg:mag:FindL} Найти промежуточные параметры \lstinline|L|, используя формулу \ref{eq:domain:findL}.
    \item Найти \lstinline|magnetCollUpdated|, используя формулу \ref{eq:domain:updateMk}.
    \item Найти сонаправленный вектор для \lstinline|magnetCollUpdated| и записать его в \lstinline|magnetColl|.
    \item Записать значение \lstinline|calcErr|, используя формулу \ref{eq:domain:findError}.
    \item Сравнить значение с \lstinline|CALIB_MIN_CALC_ERROR|, если значение меньше, то перейти дальше, иначе перейти к шагу \ref{i:alg:mag:FindL}.
    \item Записать значения от 0-8 вектора \lstinline|L| в матрицу \lstinline|softIronScales|.
    \item Записать значения от 9-11 вектора \lstinline|L| в матрицу \lstinline|hardIronScales|.
    \item Вычислить CRC-код параметров \lstinline|hardIronScales| и \lstinline|softIronScales|, записанных в одну структуру \lstinline|calibParams|.
    \item Проверить валидность вычисленного кода, если код не равен \lstinline|BL_CRC_INVALID|, то перейти дальше, иначе перейти к шагу \ref{i:alg:mag:846}.
    \item Записать в структуру \lstinline|calibParams| в поле \lstinline|checkSum| вычисленный код.
    \item Проверить выходное значение процедуры \lstinline|businessLayer_flashReadyToOperate|, если значение истинно то перейти дальше, иначе переключить контекст.
    \item Сбросить флаг состояния обратного вызова флэш памяти \lstinline|flashCb.flashIsReady|.
    \item Вызвать процедуру \lstinline|businessLayer_flashWriteAsync|, с данными \lstinline|calibParams| и адресом \lstinline|parametersAddress| из структуры дескриптора.
    \item Проверить состояние обратного вызова флэш памяти \lstinline|flashCb.flashIsReady|, если значение истинно то перейти дальше, иначе переключить контекст.
    \item \label{i:alg:mag:846} Присвоить полю \lstinline|command| дескриптора  новую команду\lstinline|CALIB_CANCEL|.
    \item Вызвать пользовательскую функцию \lstinline|cbWithoutData| с кодом \lstinline|CALIB_WARNING_INVALID_CHECKSUM|.
    \item \label{i:alg:mag:1008} Если текущее поле \lstinline|command| дескриптора равно \lstinline|CALIB_RESET_TO_DEFAULTS|, то перейти дальше, иначе перейти к шагу
    \ref{i:alg:mag:965}.
    \item Присвоить полю \lstinline|command| дескриптора  новую команду\lstinline|CALIB_APPLY|.
    \item Скопировать матрицу \lstinline|calib_softIronDef| в матрицу \lstinline|softIronScales|.
    \item Скопировать матрицу \lstinline|calib_hardIronDef| в матрицу \lstinline|hardIronScales|.
    \item Вычислить CRC-код параметров \lstinline|hardIronScales| и \lstinline|softIronScales|, записанных в одну структуру \lstinline|calibParams|.
    \item Проверить валидность вычисленного кода, если код не равен \lstinline|BL_CRC_INVALID|, то перейти дальше, иначе перейти к шагу \ref{i:alg:mag:759}.
    \item Записать в структуру \lstinline|calibParams| в поле \lstinline|checkSum| вычисленный код.
    \item Проверить выходное значение процедуры \lstinline|businessLayer_flashReadyToOperate|, если значение истинно то перейти дальше, иначе переключить контекст.
    \item Сбросить флаг состояния обратного вызова флэш памяти \lstinline|flashCb.flashIsReady|.
    \item Вызвать пользовательскую функцию \lstinline|cbWithoutData| с кодом \lstinline|BL_CALIB_MANAGER_OK|.
    \item \label{i:alg:mag:759} Вызвать пользовательскую функцию \lstinline|cbWithoutData| с кодом \lstinline|CALIB_WARNING_INVALID_CHECKSUM|.
    \item Перезапустить поток.
    \item Вызвать процедуру \lstinline|businessLayer_flashWriteAsync|, с данными \lstinline|calibParams| и адресом \lstinline|parametersAddress| из структуры дескриптора.
    \item Проверить состояние обратного вызова флэш памяти \lstinline|flashCb.flashIsReady|, если значение истинно то перейти дальше, иначе переключить контекст.
    \item \label{i:alg:mag:967} Если текущее поле \lstinline|readParametersRequest| дескриптора равно \lstinline|CALIB_REQUEST_FOR_PARAMETERS|, то перейти дальше, иначе перейти к шагу
    \ref{i:alg:mag:965}. 
    \item Проверить выходное значение процедуры \lstinline|businessLayer_flashReadyToOperate|, если значение истинно то перейти дальше, иначе переключить контекст.
    \item Сбросить флаг состояния обратного вызова флэш памяти \lstinline|flashCb.flashIsReady|.
    \item Вызвать процедуру чтения флэш памяти \lstinline|businessLayer_flashReadAsync|, с переданным параметром \lstinline|calibParams| и размером данных, и присвоить выходное значение функции переменной \lstinline|flashRetCode|.
    \item Если текущее значение \lstinline|flashRetCode| не равно \lstinline|BL_FLASH_RETURN_CODE_OK|, то перейти дальше, иначе перейти к шагу
    \ref{i:alg:mag:899}.
    \item Проверить состояние обратного вызова флэш памяти \lstinline|flashCb.flashIsReady|, если значение истинно то перейти дальше, иначе переключить контекст.
    \item Проверить состояние ошибок обратного вызова флэш памяти \lstinline|flashCb.error|, если значение истинно то перейти дальше, иначе перейти к шагу \ref{i:alg:mag:899}.
    \item Присвоить полю \lstinline|readParametersRequest| дескриптора значение \lstinline|CALIB_PARAMETERS_WAS_READ_SUCCESS|.
    \item Вычислить CRC-код прочитанных данных.
    \item Сравнить прочитанный CRC-код с вычисленным. Если условие истинно перейти к шагу \ref{i:alg:mag:974}, иначе перейти к шагу \ref{i:alg:mag:899}.
    \item \label{i:alg:mag:899} Установить поле \lstinline|readParametersRequest| дескриптора равным \lstinline|CALIBRATION_PARAMETERS_WAS_READ_FAIL|, и перейти к шагу \ref{i:alg:mag:971}.
    \item \label{i:alg:mag:971} Скопировать матрицу \lstinline|calib_softIronDef| в матрицу \lstinline|softIronScales|.
    \item Скопировать матрицу \lstinline|calib_hardIronDef| в матрицу \lstinline|hardIronScales|.
    \item Перейти к шагу \ref{i:alg:mag:974}.
\end{enumerate_step}

\subsection{Алгоритм калибровки акселерометра}

Описываемый алгоритм отвечает за калибровку акселерометра и описывает полный цикл работы модуля калибровки акселерометра: от инициализации и вычитки параметров до
вычисления калибровки и уточнения данных. 

Распишем данный алгоритм по шагам:
\begin{enumerate_step}
    \item Начало алгоритма.
    \item Проверка дескриптора, что он не равен невалидному адресу \lstinline|BL_CALIB_INVALID_HANDLE|, если равен, то выйти с ошибкой
    \lstinline|BL_CALIB_ERROR_INVALID_HANDLE|.
    \item Создание массива \lstinline|calibrateData| для хранения откалиброванных данных.
    \item Создание массива \lstinline|inputMeasurements| для хранения входных данных.
    \item Запуск потока вызовом \lstinline|PT_BEGIN|.
    \item \label{i:alg:acc:850} Проверить поле структуры обратного вызова дескриптора \lstinline|memsCb.memsWasRead|, на то были ли получены новые данные от драйвера акселерометра, 
    если нет, то переключить контекст на другие потоки, иначе выполнить следующий шаг.
    \item \label{i:alg:acc:857} Проверка существует ли запрос на чтение параметров из флэш памяти. Если поле \lstinline|readParametersRequest| дескриптора равно \lstinline|CALIB_REQUEST_FOR_PARAMETERS|,
    то перейти к следующему шагу, иначе перейти к шагу \ref{i:alg:acc:866}.
    \item Проверить возможность записи во флэш память при помощи процедуры \lstinline|businessLayer_flashReadyToOperate|, если выходное значение истинно то перейти дальше, иначе переключить контекст.
    \item Сбросить флаг состояния обратного вызова флэш памяти \lstinline|flashCb.flashIsReady|.
    \item Вызвать процедуру чтения флэш памяти \lstinline|businessLayer_flashReadAsync|, с переданным параметром \lstinline|calibParams| и размером данных, и присвоить выходное значение функции переменной \lstinline|flashRetCode|.
    \item Если текущее значение \lstinline|flashRetCode| не равно \lstinline|BL_FLASH_RETURN_CODE_OK|, то перейти дальше, иначе перейти к шагу
    \ref{i:alg:acc:568}.
    \item Проверить состояние обратного вызова флэш памяти \lstinline|flashCb.flashIsReady|, если значение истинно то перейти дальше, иначе переключить контекст.
    \item Проверить состояние ошибок обратного вызова флэш памяти \lstinline|flashCb.error|, если значение истинно то перейти дальше, иначе перейти к шагу \ref{i:alg:acc:568}.
    \item Присвоить полю \lstinline|readParametersRequest| дескриптора значение \lstinline|CALIB_PARAMETERS_WAS_READ_SUCCESS|.
    \item Вычислить CRC-код прочитанных данных.
    \item Сравнить прочитанный CRC-код с вычисленным. Если условие истинно перейти к шагу \ref{i:alg:acc:866}, иначе перейти к шагу \ref{i:alg:acc:568}.
    \item \label{i:alg:acc:568} Установить поле \lstinline|readParametersRequest| дескриптора равным \lstinline|CALIBRATION_PARAMETERS_WAS_READ_FAIL|, и перейти к шагу \ref{i:alg:acc:863}.
    \item \label{i:alg:acc:863} Скопировать матрицу \lstinline|scaleMatrixDef| в матрицу \lstinline|scaleMatrix|.
    
    \item \label{i:alg:acc:866} Если текущее поле \lstinline|command| дескриптора равно \lstinline|CALIB_APPLY|, то перейти дальше, иначе перейти к шагу
    \ref{i:alg:acc:879}.
    \item Перемножить матрицы \lstinline|inputMeasurements| и \lstinline|scaleMatrix|, выходной результат записать в \lstinline|calibrateData|.
    \item Вызвать функцию пользователя \lstinline|cbWithData| со статусом \lstinline|BL_CALIB_MANAGER_OK| и данными \lstinline|calibrateData|.
    \item \label{i:alg:acc:879} Если текущее поле \lstinline|command| дескриптора равно \lstinline|CALIB_RESET|, то перейти дальше, иначе перейти к шагу
    \ref{i:alg:acc:922}.
    \item Скопировать матрицу \lstinline|scaleMatrixDef| в матрицу \lstinline|scaleMatrix|.
    \item Вычислить CRC-код матрицы \lstinline|scaleMatrix|.
    \item Проверить валидность вычисленного кода, если код равен \lstinline|BL_CRC_INVALID|, то перейти дальше, иначе перейти к шагу \ref{i:alg:acc:891}.
    \item Вызвать пользовательскую функцию \lstinline|cbWithoutData| с кодом \lstinline|CALIB_WARNING_INVALID_CHECKSUM|.
    \item \label{i:alg:acc:891} Записать матрицу и CRC-код в структуру \lstinline|calibrationParams|.
    \item Проверить возможность записи во флэш память при помощи процедуры \lstinline|businessLayer_flashReadyToOperate|, если выходное значение истинно то перейти дальше, иначе переключить контекст.
    \item Если текущее значение \lstinline|flashRetCode| не равно \lstinline|BL_FLASH_RETURN_CODE_OK|, то перейти дальше, иначе перейти к шагу
    \ref{i:alg:acc:907}.
    \item Присвоить полю \lstinline|command| дескриптора  новую команду\lstinline|CALIB_APPLY|.
    \item \label{i:alg:acc:903} Вызвать пользовательскую функцию \lstinline|cbWithoutData| с кодом \lstinline|CALIB_WARNING_INVALID_CHECKSUM|.
    \item \label{i:alg:acc:907} Проверить состояние обратного вызова флэш памяти \lstinline|flashCb.flashIsReady|, если значение истинно то перейти дальше, иначе переключить контекст.
    \item Вызвать пользовательскую функцию \lstinline|cbWithoutData| с кодом \lstinline|BL_CALIB_MANAGER_OK|.
    
    \item \label{i:alg:acc:922} Если текущее поле \lstinline|command| дескриптора равно \lstinline|CALIB_RECALIBRATE|, то перейти дальше, иначе перейти к шагу
    \ref{i:alg:acc:985}.
    \item \label{i:alg:acc:924} Если текущее поле \lstinline|state| дескриптора равно \lstinline|CALIB_POS_FINISH|, то перейти дальше, иначе перейти к шагу
    \ref{i:alg:acc:973}.
    \item Транспонировать матрицу \lstinline|rawData| полученных данных по всем осям, результат записать в \lstinline|tRawDataMat|.
    \item Перемножить матрицы \lstinline|tRawDataMat| и \lstinline|rawData|. Записать промежуточный результат в \lstinline|tXMatProduct|.
    \item Найти обратную матрицу, для матрицы \lstinline|tXMatProduct|, результат записать в \lstinline|tRawDataMat|.
    \item Перемножить матрицы \lstinline|invProduct| и \lstinline|tRawDataMat|, результат записать в \lstinline|intrmMultProduct|.
    \item Перемножить матрицы \lstinline|intrmMultProduct| и \lstinline|positionExpectedAcceleration|, результат записать в \lstinline|scaleMatrix|.
    \item Присвоить полю \lstinline|command| дескриптора  новую команду \lstinline|CALIB_CANCEL|.
    \item Вычислить CRC-код полученной матрицы \lstinline|scaleMatrix|.
    \item Проверить валидность вычисленного кода, если код равен \lstinline|BL_CRC_INVALID|, то перейти дальше, иначе перейти к шагу \ref{i:alg:acc:932}.
    \item Вызвать пользовательскую функцию \lstinline|cbWithoutData| с кодом \lstinline|CALIB_WARNING_INVALID_CHECKSUM|.
    \item \label{i:alg:acc:932} Записать матрицу и CRC-код в структуру \lstinline|calibrationParams|.
    \item Проверить возможность записи во флэш память при помощи процедуры \lstinline|businessLayer_flashReadyToOperate|, если выходное значение истинно то перейти дальше, иначе переключить контекст.
    \item Если текущее значение \lstinline|flashRetCode| не равно \lstinline|BL_FLASH_RETURN_CODE_OK|, то перейти дальше, иначе перейти к шагу
    \ref{i:alg:acc:953}.
    \item Присвоить полю \lstinline|command| дескриптора  новую команду \lstinline|CALIB_APPLY|.
    \item Вызвать пользовательскую функцию \lstinline|cbWithoutData| с кодом \lstinline|CALIB_WARNING_INVALID_CHECKSUM|.
    \item \label{i:alg:acc:953} Проверить состояние обратного вызова флэш памяти \lstinline|flashCb.flashIsReady|, если значение истинно то перейти дальше, иначе переключить контекст.
    \item Присвоить полю \lstinline|state| дескриптора новое состояние \lstinline|CALIB_FINISHED|.
    \item Вызвать пользовательскую функцию \lstinline|cbWithState| с кодом \lstinline|BL_CALIB_MANAGER_OK| и состоянием из поля \lstinline|state| дескриптора.

    \item \label{i:alg:acc:973} Если текущее поле \lstinline|state| дескриптора равно \lstinline|CALIB_FIRST_POS|, то перейти дальше, иначе перейти к шагу
    \ref{i:alg:acc:981}.
    \item Перезапустить поток, перейти к шагу \ref{i:alg:acc:850}.
    \item \label{i:alg:acc:981} Если текущее поле \lstinline|command| дескриптора равно \lstinline|CALIB_SET_POS|, то перейти дальше, иначе перейти к шагу
    \ref{i:alg:acc:850}.
    \item Вызвать пользовательскую функцию \lstinline|cbWithState| с кодом \lstinline|CALIB_ERROR_COMMAND| и состоянием из поля \lstinline|state| дескриптора.

    \item \label{i:alg:acc:985} Если текущее поле \lstinline|command| дескриптора равно \lstinline|CALIB_SET_POS|, то перейти дальше, иначе перейти к шагу
    \ref{i:alg:acc:1010}.
    \item \label{i:alg:acc:988} Если текущее поле \lstinline|measurementCount| дескриптора меньше \lstinline|CALIB_AMOUNT_PACK| и поле \lstinline|state| равно
    \lstinline|CALIB_LAST_POS|, то перейти дальше, иначе перейти к шагу \ref{i:alg:acc:996}.
    \item Присвоить полю \lstinline|command| дескриптора  новую команду \lstinline|CALIB_RECALIBRATE|.
    \item Присвоить полю \lstinline|state| дескриптора новое состояние \lstinline|CALIB_POS_FINISH|.
    \item Вызвать пользовательскую функцию \lstinline|cbWithState| с кодом \lstinline|BL_CALIB_MANAGER_OK| и состоянием из поля \lstinline|state| дескриптора.
    \item \label{i:alg:acc:996} Если текущее поле \lstinline|measurementCount| дескриптора меньше \lstinline|CALIB_AMOUNT_PACK|, то перейти дальше, иначе перейти к шагу \ref{i:alg:acc:1003}.
    \item Увеличить поле \lstinline|state| дескриптора на единицу.
    \item Вызвать пользовательскую функцию \lstinline|cbWithState| с кодом \lstinline|BL_CALIB_MANAGER_OK| и состоянием из поля \lstinline|state| дескриптора.
    \item Присвоить полю \lstinline|command| дескриптора новую команду\lstinline|CALIB_WAIT_NEW_POS|.
    \item Перейти к шагу \ref{i:alg:acc:850}.
    \item \label{i:alg:acc:1003} Увеличить смещение \lstinline|positionInStorage| на количество полученных данных.
    \item Присвоить массиву \lstinline|eachPositionData| полученные данные.
    \item Увеличить счетчик \lstinline|measurementCount| на единицу.

    \item \label{i:alg:acc:1010} Если текущее поле \lstinline|command| дескриптора равно \lstinline|CALIB_CANCEL|, то перейти дальше, иначе перейти к шагу
    \ref{i:alg:acc:1020}.
    \item Сбросить счетчик \lstinline|measurementCount|.
    \item Присвоить полю \lstinline|command| дескриптора  новую команду \lstinline|CALIB_APPLY|.
    \item Присвоить полю \lstinline|state| дескриптора новое состояние \lstinline|CALIB_POS_FINISHED|.
    \item Вызвать пользовательскую функцию \lstinline|cbWithoutData| с кодом \lstinline|BL_CALIB_MANAGER_OK|.
    \item Вызвать пользовательскую функцию \lstinline|cbWithState| с кодом \lstinline|BL_CALIB_MANAGER_CANCEL| и состоянием из поля \lstinline|state| дескриптора.
    \item \label{i:alg:acc:1020} Если текущее поле \lstinline|command| дескриптора равно \lstinline|CALIB_WAIT_NEW_POS|, то перейти к шагу
    \ref{i:alg:acc:850}.
\end{enumerate_step}

\subsection{Алгоритм калибровки гироскопа}

Описываемый алгоритм отвечает за калибровку гироскопа и описывает полный цикл работы модуля калибровки акселерометра: от инициализации и вычитки параметров до
вычисления калибровки и уточнения данных. Данный алгоритм не был описан в разделе \ref{sec:domain}, полное понимание принципа калибровки будет дано в данном подразделе.

\begin{enumerate_step}
    \item Начало алгоритма.
    \item Проверка дескриптора, что он не равен невалидному адресу \lstinline|BL_CALIB_INVALID_HANDLE|, если равен, то выйти с ошибкой
    \lstinline|BL_CALIB_ERROR_INVALID_HANDLE|.
    \item Запуск потока вызовом \lstinline|PT_BEGIN|.
    \item \label{i:alg:gyro:551} Проверить поле структуры обратного вызова дескриптора \lstinline|memsCb.memsWasRead|, на то были ли получены новые данные от драйвера акселерометра, 
    если нет, то переключить контекст на другие потоки, иначе выполнить следующий шаг.
    \item \label{i:alg:gyro:557} Проверка существует ли запрос на чтение параметров из флэш памяти. Если поле \lstinline|readParametersRequest| дескриптора равно \lstinline|CALIB_REQUEST_FOR_PARAMETERS|,
    то перейти к следующему шагу, иначе перейти к шагу \ref{i:alg:gyro:566}.
    \item Проверить возможность записи во флэш память при помощи процедуры \lstinline|businessLayer_flashReadyToOperate|, если выходное значение истинно то перейти дальше, иначе переключить контекст.
    \item Сбросить флаг состояния обратного вызова флэш памяти \lstinline|flashCb.flashIsReady|.
    \item Вызвать процедуру чтения флэш памяти \lstinline|businessLayer_flashReadAsync|, с переданным параметром \lstinline|calibParams| и размером данных, и присвоить выходное значение функции переменной \lstinline|flashRetCode|.
    \item Если текущее значение \lstinline|flashRetCode| не равно \lstinline|BL_FLASH_RETURN_CODE_OK|, то перейти дальше, иначе перейти к шагу
    \ref{i:alg:gyro:405}.
    \item Установить поле \lstinline|readParametersRequest| дескриптора равным \lstinline|CALIBRATION_PARAMETERS_WAS_READ_FAIL|, и перейти к шагу \ref{i:alg:gyro:551}.
    \item \label{i:alg:gyro:405}  Проверить состояние обратного вызова флэш памяти \lstinline|flashCb.flashIsReady|, если значение истинно то перейти дальше, иначе переключить контекст.
    \item Проверить состояние ошибок обратного вызова флэш памяти \lstinline|flashCb.error|, если значение истинно то перейти дальше, иначе перейти к шагу \ref{i:alg:gyro:551}.
    \item Присвоить полю \lstinline|readParametersRequest| дескриптора значение \lstinline|CALIB_PARAMETERS_WAS_READ_SUCCESS|.
    \item Вычислить CRC-код прочитанных данных.
    % TODO: переход 551 во многих местах неверный исправить
    \item Сравнить прочитанный CRC-код с вычисленным. Если условие истинно перейти к шагу \ref{i:alg:gyro:551}, иначе перейти к шагу \ref{i:alg:gyro:551}.
    
    \item \label{i:alg:gyro:566} Если текущее поле \lstinline|command| дескриптора равно \lstinline|CALIB_APPLY|, то перейти дальше, иначе перейти к шагу
    \ref{i:alg:gyro:571}.

    \item \label{i:alg:gyro:571} Если текущее поле \lstinline|command| дескриптора равно \lstinline|CALIB_RECALIBRATE|, то перейти дальше, иначе перейти к шагу
    \ref{i:alg:gyro:599}.

    \item \label{i:alg:gyro:599} Если текущее поле \lstinline|command| дескриптора равно \lstinline|CALIB_CANCEL|, то перейти дальше, иначе перейти к шагу
    \ref{i:alg:gyro:605}.

    \item \label{i:alg:gyro:605} Если текущее поле \lstinline|command| дескриптора равно \lstinline|CALIB_RESET|, то перейти дальше, иначе перейти к шагу
    \ref{i:alg:gyro:551}.

\end{enumerate_step}

% \subsection{Алгоритм расчета ориентации и азимута}

% \subsection{Алгоритм управления калибровкой}
