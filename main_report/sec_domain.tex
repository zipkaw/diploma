\section{ОБЗОР ЛИТЕРАТУРЫ}
\label{sec:domain}

\subsection{Метод калибровки магнетометра}
% https://www.ncbi.nlm.nih.gov/pmc/articles/PMC8401862/
% https://teslabs.com/articles/magnetometer-calibration/
Калибровка магнетометра - это критически важный этап для обеспечения точности измерений магнитного поля. 
Суть этого процесса заключается в выявлении и компенсации различных видов погрешностей, которые могут 
влиять на получаемые данные. Эти погрешности делятся на две категории: статические и динамические смещения.
Статические смещения постоянны во времени и обусловлены особенностями производства датчиков, например,
неидеальностью материалов, неточностью монтажа. Среди таких помех существуют: 
\begin{itemize}
    \item смещение нуля: отклонение измеряемого значения от истинного нуля
    \item масштабный коэффициент: Неравномерность чувствительности датчика по различным осям
    \item ненулевая ортогональность: Неидеальность взаимного расположения осей чувствительности датчика
\end{itemize}

Динамические смещения переменны во времени и вызваны внешними факторами окружающей среды, например, магнитными помехами
вызванные ближайшими электронными устройствами и металлическими предметами, 
а также температурные изменения, к которым может быть чувствителен датчик.

Почти все методы калибровки используют упрощенную модель измерения трехосного магнетометра:

$$ y = Tm+h+e$$

где $h$~-- жесткие смещенияб то есть смещения вызванные материалами, которые вызывают смещения, прикрепленные к системе координат, 
такие материалы генерируют собственное  постоянное магнитное поле,
$T$~-- мягкие смещение (представляет собой матрицу 3x3) --- смещения вызванные наличием ферромагнитных материалов вблизи датчика,
эти материалы не генерируют собственное магнитное поле, а вместо этого локально изменяют существующее магнитное поле, что приводит к расхождению измерений,
$e$~-- случайный шум, вызванные особенностями механической и электрической архитектуры датчика, считается, что это последовательность белого шума.
$m$~-- истинное магнитное поле, $y$~-- искаженное магнитное поле.

Используемый в дипломном проекте метод калибровки магнетометра, называемый <<MAG.I.C.AL.>> , предполагает, что данные измерений магнитометра
лежат на эллипсоиде согласно модели измерений, то есть набор векторов, представляющий измерения магнетометра по трём осям, 
имеет вид на трехмерной плоскости в виде эллипсоида. %%TODO: add link to equitation %%. 

%TODO: поправить формулу%
Для алгоритма требуется проинициализировать вектора магнитного поля:

$$ mk = yk/|yk| $$

где $K$ количество измерений.

Алгоритм использует другую форму модели измерений записанную как:

$$ Y = LG + E $$

где
$$ Y = [y1..yk] $$,
$$ L = [T h] $$,
$$ G = [m1..mk][1..1] $$,
$$ E = [e1..ek] $$,

$L$ ищется при помощи метода наименьших квадратов, используя формулу:

$$ L=YG^T(GG^T)^-1$$

По полученным параметрам $T$ и $h$ извлеченных из $L$, обновим вектор магнитного поля:
$$ mk = m~k/|m~k| $$

, где $mk$ найдено из формулы:  
$$ m~k = T^-1(yk - h) $$

Далее вычисленный вектор оценивается при помощи функции:

$$ J(T, h) = ... $$

Если оценка не достигла минимального значения, то $G$ обновляется вычисленным вектором $m~k$ и снова вычисляется $T$ и $h$.
Обобщенно алгоритм можно представить так:
% TODO:Добавить ссылки на формулы %.
% TODO:Дизобразить в виде блок схемы%.
\begin{enumerate}
    \item Инициализация mk.
    \item Расчет L.
    \item извлечение T, h из L.
    \item Обновление вектора mk из формулы 
    \item Оценка полученного вектора mk.
    \item Повтор шага 2-5
\end{enumerate}

\subsubsection{Сравнение актуальных алгоритмов калибровки магнетометра}

Помимо выбранного алгоритма калибровки магнетометра существуют также другие способы вычислить параметры T и h. 
В статье %ссыль%%
приводится сравнение известных алгоритмов по их точности, надежности и среднему времени выполнения, степени простоты.
Параметр точности описывается как отклонение полученных параметров h и T от истинных h и T.

\begin{table}[ht]
    \caption{Сравнение алгоритмов калибровки магнетометра}
    \label{table:domain:magnet_calib_comp}
    \begin{tabular}{| >{\raggedright}m{0.397\textwidth}
                    | >{\raggedright\arraybackslash}m{0.55\textwidth}|
                    | >{\raggedright\arraybackslash}m{0.55\textwidth}|
                    | >{\raggedright\arraybackslash}m{0.55\textwidth}|
                    | >{\raggedright\arraybackslash}m{0.55\textwidth}|}
        \hline
        \centering Точность & 
        \centering\arraybackslash Надежность &
        \centering\arraybackslash Время выполнения &
        \centering\arraybackslash Степень простоты
        \\

        \hline
        ФФ & ФФ & AA & AA \\

        \hline
    \end{tabular}
\end{table}

%TODO: Дописать сравнение и заполнить обобщенную таблицу %

\subsection{Метод калибровки акселерометра и гироскопа}

На измерения гироскопа и акселерометра влияют такие параметры как чувствительность датчика и смещение относительно нуля.
На такие параметры в датчике обычно влияет температура.
Для описания модели измерений можно воспользоваться формулой %% TODO: ссылка из магнетометра %%
, которую можно упростить до вида:

$$ y = X*m $$
, где X это матрица состоящая из T и h.
В формуле T описывает параметр чувствительности, а параметр h смещение относительно нуля.
Для калибровки гироскопа в разрабатываемом модуле не предусмотрен поиск параметра T в силу того, 
что нужно разрабатывать отдельный калибровочный стенд способный вращать устройство с определенной 
угловой скоростью относительно каждой оси трехмерной системы координат. Но описанный ниже алгоритм подходит для двух типов датчиков, 
только вместо ускорения свободного падения относительно каждой оси нужно заменить на угловую скорость.

Для алгоритма потребуются начальные входные данные от неоткалиброванного устройства в шести положениях относительно датчика.
\begin{enumerate}
    \item Положение осью X вниз.
    \item Положение осью X вверх.
    \item Положение осью Y вниз.
    \item Положение осью Y вверх.
    \item Положение осью Z вниз.
    \item Положение осью Z вверх.
\end{enumerate}

Для каждого положения определим эталонный вектор свободного ускорения падения Y, например для положения осью X вверх, вектор Y имеет вид:

$$ Yxup = [1 0 0] $$

Для положения осью Z вниз, эталонный вектор будет иметь вид:

$$ Yzdown = [0 0 -1] $$

Также для каждого положения нужно получить набор неоткалиброванных данных. 
К вектору неоткалиброванных данных нужно добавить еще один четвертый элемент 1, теперь вектор входных данных имеет вид:

$$ m = [mx my mz 1]$$

Далее нужно сформировать полный набор данных, как эталонных так и выходных, матрицы будут иметь вид:

$$ 
Y^T = [Y1...Y6],
m^T = [m1..m6]
$$

Следующим шагом используя метод наименьших квадратов можно определить калибровочные параметры:

$$  Мне лень писать формулу $$

Данный алгоритм достаточно популярен за счет простоты математичеких расчетов, недостатком можно выделить то что 
для алгоритма требуется большой набор данных состоящий из набора векторов для каждой оси, в микроконтроллерах ограниченных небольшим 
количеством памяти требуется оптимизировать алгоритм. 

\subsection{Фильтр Маджвика}

Чтобы оценить ориентацию тела в пространстве, нужно для начала выбрать какие-то параметры, 
которые в совокупности однозначно определяют эту ориентацию, т.е. по сути ориентацию связанной 
системы координат $xyz$ относительно условно неподвижной системы — например, географической системы NED (North, East, Down). 
Затем нужно составить кинематические уравнения, т.е. выразить скорость изменения этих параметров через угловую скорость от 
гироскопов. Наконец, нужно ввести в расчёт векторные измерения от акселерометра.

Одним из способов представить ориентацию тела в пространстве является квантерион, который используется фильтром Маджвика в расчетах.
Это четырехмерное комплексное число $\mathbf{q}=q_{0}+q_{1}\mathbf{i}+q_{2}\mathbf{j}+q_{3}\mathbf{k}$, которое может быть использовано 
для представления ориентации тела в трехмерном пространстве,
Описать связь квантериона с теоремой Эйлера можно через формулу: 

%https://habr.com/ru/articles/255661/%
%https://habrastorage.org/r/w1560/files/b65/424/f94/b65424f943e5432185bdc29631cd5d37.png%

Целью фильтра Мэджвика и похожих фильтров, является компенсация дрейфа гироскопа по времени и по температуре.
Используя квантерионы, которые позволяют представить данные акселерометра и магнетометра пригодные для решения задачи оптимизации 
методом градиентного спуска для того чтобы выявить погрешность измерений гироскопа связанных с дрейфом. Это эффективная в 
вычислительном отношении альтернатива фильтру Калмана, что делает ее подходящей для приложений реального времени во встроенных 
системах с ограниченными ресурсами. 

Преимуществами данного фильтра являются:

\begin{enumerate}
    \item дешевизна по вычислительным ресурсам — 277 простых арифметических операций каждое обновление фильтра;
    \item эффективность при низких частотах дискретизации (например 10 Гц);
\end{enumerate}

Недостатком фильра является множество параметров к которым он восприимчив, поэтому его настройка должна быть очень тщательная.

Фильтр состоит из нескольких блоков представленных на рисунке~\ref{pic::domain::madgwick}.
Первый блок это прием данных и их нормализация для обеспечения одинакового 
масштаба поступаемых с датчиков данных.
Второй блок вычисляет ошибку между прогнозируемыми (на основе предполагаемой ориентации) и поступающими данными, 
эта ошибка используется следующим блоком.
Следующий, ключевой, в фильтре блок --- это реализация градиентного спуска, который 
использует итерационный процесс для минимизации ошибки между прогнозируемыми и 
измеренными значениями датчика. Алгоритм вычисляет градиент функции ошибок и корректирует
предполагаемую ориентацию (представленную в виде кватерниона) для достижения минимальной ошибки.
Все это вместе компенсирует дрейф гироскопа, который накапливается со временем.

\begin{figure}[ht]
    \centering
    \includegraphics[width=1.0\linewidth]{madgwick}
    \caption{Блок-схема фильтра Мэджвика}
    \label{pic::domain::madgwick}
\end{figure}

\subsection{Метод расчета азимута с учетом наклона устройства}

После калибровки данных, полученных от магнетометра, следуют расчеты азимута в сторону сервера.
Для прибора который находится в свободном положении это может быть проблемой с учетом того, что
он может быть наклонен и угол проекции вектора измерений магнетометра может не совпасть с настоящим азимутом.
Для этого требуется поправка азимута с учетом наклона устройства. Принцип заключается в использовании 
поворотной матрицы, которая поворачивает ось измерений согласно наклону устройства. Наклон устройства 
считывается с датчиков акселерометра и гироскопа.

Для расчета азимута потребуется только две оси магнетометра $x$~,$y$, ось $z$ в расчетах будет избыточной.
Чтобы представить азимут в радианах нужно воспользоваться формулой:

$$ azimuth = atan(Hx/Hy)$$

, где $Hx$,$Hy$ --- компоненты вектора измерения магнетометра. 

Также на компас действует магнитное наклонение Земли --- угол, на который отклоняется стрелка под действием 
магнитного поля Земли в вертикальной плоскости. Для уточнения азимута нужно использовать магнитную модель Земли.
Достоверная модель показана на рисунке~\ref{pic::domain::wmm}.

% https://www.ncei.noaa.gov/products/world-magnetic-model
\begin{figure}[ht]
    \centering
    \includegraphics[width=1.0\linewidth]{wmm}
    \caption{Магнитная модель Земли}
    \label{pic::domain::wmm}
\end{figure}

%% Если будет мало то можно добавить расчет бих фильтра%%
\subsection{БИХ-фильтр}

Стандарт \iecStd\ описывает свод правил для организации событийного протокола
передачи данных. Область применения стандарта -- системы связи внутри подстанции.
В набор стандартов входят стандарт по одноранговой связи и связи клиент/сервер,
стандарт по структуре и конфигурации подстанции, стандарт по методике испытаний,
стандарт экологических требований и другие~\cite{iec_description}.

\begin{figure}[ht]
    \centering
    \includegraphics[width=1.0\linewidth]{msgTypes}
    \caption{Основные протоколы \iecStd}
    \label{pic::domain::msg_types}
\end{figure}

Целью стандарта \iecStdRef81\ является предоставление подробных
инструкций и
спецификаций в отношении механизмов и правил, необходимых для реализации сервисов,
объектов и алгоритмов, указанных в других стандартах \iec, таких как
\iecStd-7-2, \iecStd-7-3 и \iecStd-7-4~\cite{IEC61850_7_2, IEC61850_7_3,
IEC61850_7_4}, с использованием спецификаций производственных сообщений,
SNTP и других прикладных протоколов, как показано
на рисунке~\ref{pic::domain::msg_types}.
В стандарте описаны следующие типы сообщений:

\begin{itemize}
    \item тип 1 -- быстрые сообщения;
    \item тип 1А -- сообщения отключения;
    \item тип 2 -- сообщения со средней скоростью;
    \item тип 3 -- сообщения с низкой скоростью;
    \item тип 4 -- сообщения с необработанными данными;
    \item тип 5 -- функции передачи файлов;
    \item тип 6 -- сообщения синхронизации времени.
\end{itemize}

\nomenclaturex{SNTP}{Simple Network Time Protocol}{упрощенная реализация протокола синхронизации времени NTP}

Сообщения типов 1 и 1A сопоставляются с различными типами EtherType
(поле в кадре Ethernet) для оптимизации декодирования полученных сообщений.

\subsection{Соответствие с моделью OSI}

\nomenclaturex{OSI}{Open Systems Interconnection}{взаимосвязь открытых систем}

Эталонная модель OSI детализирует модель, основанную на концепции многоуровневой
коммуникационной функциональности. Модель конкретизирует семь уровней и выделяет
функциональные требования для каждого уровня, чтобы создать надежную систему связи.
Она не определяет протоколы, которые должны использоваться для достижения
функциональности, и не ограничивает решение одним набором протоколов.

\begin{figure}[ht]
    \centering
    \includegraphics[width=.5\linewidth]{osiModel}
    \caption{Эталонная модель и профили OSI}
    \label{pic::domain::osi_model}
\end{figure}

В модели OSI используется два основных профиля: A-профиль -- профиль приложения
и T-профиль -- транспортный профиль. Профили сопоставлены с уровнями модели,
как изображено на рисунке~\ref{pic::domain::osi_model}. А-профиль OSI --
это набор спецификаций и соглашений, которые относятся к трем верхним уровням
эталонной модели OSI (уровень приложения, уровень представления и сессионный
уровень). Т-профиль относится к четырем нижним уровням модели
(транспортный уровень, сетевой, канальный и физический). Можно использовать
различные комбинации А- и Т-профилей, чтобы обеспечить определенные виды информации
и сервисов, подлежащих обмену. Сервисы представлены в четырех различных комбинациях
А- и Т-профилей. Они используются для следующих служб:

\begin{itemize}
    \item службы клиент/сервер, изображенные на рисунке~\ref{pic::domain::msg_types} как набор протоколов MMS;
    \item службы управления GOOSE/GSE;
    \item сервисы GSSE;
    \item сервисы синхронизации времени, изображенные на рисунке~\ref{pic::domain::msg_types} как SNTP.
\end{itemize}

\nomenclaturex{GSSE}{Generic Substation State Event}{обобщенное событие состояния подстанции}
\nomenclaturex{GSE}{Generic Substation Event}{обобщенное событие подстанции}

\subsection{Службы управления GOOSE/GSE}

Коммуникационный профиль GSE рекомендуется использовать для любой реализации,
утверждающей соответствие стандарту \iecStdRef81\ и заявляющей
о поддержке одного из сервисов \iecStdRef72, показанных
в таблице~\ref{table:domain:management_services}.

\begin{table}[ht]
    \caption{Сервисы, требующие управления коммуникационных профилей GSE и GOOSE}
    \label{table:domain:management_services}
    \begin{tabular}{| >{\raggedright}m{0.397\textwidth}
                    | >{\raggedright\arraybackslash}m{0.55\textwidth}|}
        \hline
        \centering Модель & \centering\arraybackslash Сервис \iecStdRef72 \\

        \hline
        GSE & GetReference \\

         & GetGOOSEElementNumber \\

         & SendGOOSEMessage \\

        \hline
    \end{tabular}
\end{table}

В таблице~\ref{table:domain:gse_management} показаны службы и протоколы A-профиля
для общения GOOSE и управления GSE.

\begin{table}[ht]
    \caption{Службы и протоколы A-профиля для общения GOOSE и управления GSE}
    \label{table:domain:gse_management}
    \begin{tabular}{| >{\raggedright}m{0.19\textwidth}
                    | >{\raggedright}m{0.32\textwidth}
                    | >{\raggedright\arraybackslash}m{0.41\textwidth}|}
        \hline
        \centering Уровень модели OSI &
        \centering Имя &
        \centering\arraybackslash Спецификация службы и протокола \\

        \hline
        Прикладной & Протокол GOOSE/GSE & \iecStdRef81, приложение~А \\

        \hline
        Представления & Абстрактный синтаксис & \centering\arraybackslash --- \\

        \hline
        Сессионный & & \\

        \hline
    \end{tabular}
\end{table}

T-профиль для сервисов GSE и GOOSE должен соответствовать
таблице~\ref{table:domain:t_profile_for_gse}.

На базе таблиц~\ref{table:domain:gse_management}
и~\ref{table:domain:t_profile_for_gse} были приняты соглашения о реализации служб,
представленных в таблице~\ref{table:domain:management_services}. Кодирование уровня
представления должно соответствовать основным правилам кодирования. Все PDU должны
отправляться и приниматься с использованием службы T-DATA. Адрес назначения сервиса
T-DATA для сообщения GOOSE должен содержать мультикаст MAC-адрес. Адрес источника
T-DATA для сообщения GOOSE должен содержать юникаст MAC-адрес. Адрес назначения
и адрес источника T-DATA для сообщений управления GSE должны содержать юникаст
MAC-адрес.

\nomenclaturex{PDU}{Protocol Data Unit}{блок данных протокола}

\begin{table}[ht]
    \caption{T-профиль для сервисов GSE и GOOSE}
    \label{table:domain:t_profile_for_gse}
    \begin{tabular}{| >{\raggedright}m{0.20\textwidth}
                    | >{\raggedright}m{0.35\textwidth}
                    | >{\raggedright\arraybackslash}m{0.37\textwidth}|}
        \hline
        \centering Уровень модели OSI &
        \centering Имя &
        \centering\arraybackslash Спецификация службы и протокола \\

        \hline
        Транспортный & & \\

        \hline
        Сетевой & & \\

        \hline
        \multirow{2}{0.20\textwidth}{Резервирование канала связи} & Кольцо HSR и PRP & \iec~62439-3 -- PRP или HSR \\

        \cline{2-3}
        & RSTP & \ieee~802.1D \\

        \hline
        Канальный & Маркировка приоритета / VLAN & \ieee~802.1Q \\

        \hline
        Физический (общее) & CSMA/CD & \isoIec~8802-3:2000 \\

        \hline
        % Hand fixing: https://tex.stackexchange.com/a/66599/139966
        \multirow{2}{0.20\textwidth}[-1.5em - 0.5ex]{Физический (вариант 1)}
        & 10Base-T/100Base-T
        & \isoIec~8802-3:2000 \\

        \cline{2-3}
        & Интерфейсный разъем и назначение контактов для базового интерфейса доступа ISDN
        & \isoIec~8877:1992 \\

        \hline
        % Hand fixing: https://tex.stackexchange.com/a/66599/139966
        \multirow{2}{0.20\textwidth}[-1em - 0.5ex]{Физический (вариант 2)}
        & Волоконно-оптическая система передачи 1000Base-LX
        & \isoIec~8802-2:1998, \isoIec~8802-3:2000 \\

        \cline{2-3}
        & Базовый оптоволоконный соединитель
        & \iec~60874-10-1, \iec~60874-10-2 и
        \iec~60874-10-3 \\

        \hline
    \end{tabular}
\end{table}

\nomenclaturex{PRP}{Parallel Redundancy Protocol}{протокол параллельного резервирования}
\nomenclaturex{HSR}{High-availability Seamless Redundancy}{бесперебойное резервирование с высокой доступностью}
\nomenclaturex{IEEE}{Institute of Electrical and Electronics Engineers}{институт инженеров электротехники и электроники}
\nomenclaturex{RSTP}{Rapid Spanning Tree Protocol}{ускоренный протокол STP}
\nomenclaturex{VLAN}{Virtual Local Area Network}{виртуальная локальная компьютерная сеть}
\nomenclaturex{ISDN}{Integrated Services Digital Network}{цифровая сеть интегрированных услуг}
\nomenclaturex{CSMA/CD}{Carrier Sense Multiple Access with Collision Detection}
{множественный доступ с прослушиванием несущей и обнаружением коллизий}

\subsubsection{Служба GetGoReference}

Служба GetGoReference позволяет клиенту запрашивать разрешение на получение смещений
одного или нескольких элементов. В ответе возвращается набор значений,
соответствующих запрошенным ElementOffsets. Алгоритм работы сервиса GetGoReference
изображен на рисунке~\ref{pic::domain::get_go_ref_algo}.

\begin{figure}[!htb]
    \centering
    \includegraphics[width=.9\linewidth]{GetGoReference_algo}
    \caption{Алгоритм работы сервиса GetGoReference}
    \label{pic::domain::get_go_ref_algo}
\end{figure}

\nomenclaturex{MMS}{Manufacturing Message Specification}{спецификация производственного сообщения}

\begin{table}[ht]
    \caption{Параметры GetGoReference}
    \label{table:domain:get_go_ref_params}
    \begin{tabular}{| >{\raggedright}m{0.247\textwidth}
                    | >{\raggedright\arraybackslash}m{0.7\textwidth}|}
        \hline
        \centering Поле & \centering\arraybackslash Описание \\

        \hline
        Destination address & Адрес назначения должен использоваться для указания адреса, требуемого Т-профилем. \\

        \hline
        StateID & Присваиваемое клиентом значение, используемое для идентификации.
        Диапазон этого значения должен быть от $ -32767 $ до $ 32767 $. \\

        \hline
        GoCBReference & Поле типа VISIBLE\_STRING, содержащее значение,
        размер которого составляет 129 октетов. Значение должно соответствовать
        управляющему блоку GOOSE, для которого запрашивается поиск. \\

        \hline
        MemberOffsets & Это список элементов, для которых клиент запрашивает смещения. Диапазон этого значения должен быть от 0 до 512. \\

        \hline
        ConfRev & Параметр, содержащий номер версии конфигурации GoCB на момент запроса. \\

        \hline
        DatSet & Параметр, содержащий значение DataSetReference на момент запроса. \\

        \hline
        ListOfResults & Список значений запрашиваемых смещений на строки или
        соответствующий код ошибки. Отправляется с использованием сервиса T-DATA. \\

        \hline
        ErrorReason & Параметр состояния ошибки клиентского запроса. \\

        \hline
    \end{tabular}
\end{table}

Клиент присваивает идентификатор каждому запросу и включает его в качестве параметра
StateID в запрос. При получении GetGoReferenceResponse, который содержит неизвестный
StateID, клиент должен игнорировать PDU. Спецификация протокола прикладного уровня
MMS должна использоваться в качестве синтаксиса передачи для службы GetGoReference.
Сервис GetGoReference должен быть отображен, учитывая соответствие имя параметра
и синтаксиса передачи.
В таблице~\ref{table:domain:get_go_ref_params} представлено описание параметров
GetGoReference.

Все PDU управления GSE должны отправляться и приниматься с использованием службы
T-DATA.

\subsubsection{Служба GetGOOSEElementNumber}

Служба GetGOOSEElementNumber, как определено в \iecStdRef72, позволяет клиенту
запросить преобразование одной или нескольких строк в виде смещения элементов.
В ответе возвращается набор запрошенных ElementOffsets. Последовательность работы
алгоритма должна соответствовать рисунку~\ref{pic::domain::get_goose_elem_num_algo}.

\begin{figure}[ht]
    \centering
    \includegraphics[width=1.\linewidth]{GetGOOSEElementNumber_algo}
    \caption{Алгоритм работы сервиса GetGOOSEElementNumber}
    \label{pic::domain::get_goose_elem_num_algo}
\end{figure}

Клиент назначает идентификатор каждому запросу, записывая его в параметр StateID
запроса. Клиент, который получает GetGOOSEElementNumberResponse, содержащий
неизвестный StateID, должен игнорировать PDU. Сервер, заявляющий о поддержке
сервиса GOOSE Management, не являющийся сервисом GetGOOSEElementNumber,
должен возвращать GseNotSupportedPDU, если был получен GetGOOSEElementNumberRequest.
Спецификация протокола прикладного уровня MMS должна использоваться в качестве
синтаксиса передачи для службы GetGOOSEElementNumber.
Сервис должен быть отображен, учитывая соответствие имя параметра
и синтаксиса передачи.

В таблице~\ref{table:domain:get_goose_elem_number_params} представлено описание
уникальных параметров GetGOOSEElementNumber. Остальные параметры соответствуют службе
GetGoReference и описаны в таблице~\ref{table:domain:get_go_ref_params}.

\begin{table}[ht]
    \caption{Параметры GetGOOSEElementNumber}
    \label{table:domain:get_goose_elem_number_params}
    \begin{tabular}{| >{\raggedright}m{0.247\textwidth}
                    | >{\raggedright\arraybackslash}m{0.7\textwidth}|}
        \hline
        \centering Поле & \centering\arraybackslash Описание \\

        \hline
        MemberReference & Список элементов, для которых клиент запрашивает смещения. Значения NULL не допускаются. \\

        \hline
        ElementNumber & Параметр содержит значение смещения для соответствующего
        запрошенного ReferenceString или причину ошибки. \\

        \hline
        T-DATA Mapping & Все PDU управления GSE должны отправляться и приниматься
        с использованием службы T-DATA T-Profile. \\

        \hline
    \end{tabular}
\end{table}

\fixTableSectionSpace

\subsubsection{Служба SendGOOSEMessage}

Модель работы GOOSE дает возможность быстрого и надежного общесистемного распределения
значений входных и выходных данных. Она использует определенную схему повторной
передачи для достижения надлежащего уровня надежности. Когда сервер GOOSE генерирует
запрос SendGOOSEMessage, текущие значения набора данных кодируются в сообщение
GOOSE и передаются через сервис T-DATA в многоадресной рассылке.
Дополнительная надежность достигается за счет повторной передачи одних
и тех же данных с постепенным увеличением SqNum и времени повторной передачи.
Алгоритм передачи данных представлен
на рисунке~\ref{pic::domain::events_transmission_time}, где:
\begin{explanationx}
    \item $ \text{Т}_0 $ -- повторная отправка в стабильных условиях (длительное отсутствие событий);
    \item $ (\text{Т}_0) $ -- повторная передача в стабильных условиях, может быть сокращена событием;
    \item $ \text{Т}_1 $ -- кратчайшее время ретрансляции после события;
    \item $ \text{Т}_2, \text{Т}_3 $ -- время повторной передачи до достижения стабильных условий.
\end{explanationx}

\begin{figure}[ht]
    \centering
    \includegraphics[width=1.\linewidth]{eventsTransmissionTime}
    \caption{Временная развертка передачи данных}
    \label{pic::domain::events_transmission_time}
\end{figure}

Каждое сообщение содержит параметр timeAllowedToLive, который сообщает получателю
максимальное время ожидания следующей повторной передачи. Если новое сообщение
не получено в течение этого интервала времени, то сообщение считается потерянным.
Интервалы, используемые сервером GOOSE, являются настраиваемым вопросом и
информируют клиентов о времени ожидания повторной отправки. Служба SendGOOSEMessage
позволяет отправлять информацию, если она была не запрошена или не подтверждена.
Алгоритм работы службы SendGOOSEMessage представлен
на рисунке~\ref{pic::domain::send_goose_msg_algo}.

\begin{figure}[ht]
    \centering
    \includegraphics[width=.75\linewidth]{SendGOOSEMessage_algo}
    \caption{Алгоритм работы сервиса SendGOOSEMessage}
    \label{pic::domain::send_goose_msg_algo}
\end{figure}

Отправитель создает конечный автомат, как изображено на
рисунке~\ref{pic::domain::fsm_sender_goose}, для каждого
разрешенного GoCB, состоящий из четырех состояний:
\begin{itemize}
    \item данные не существуют;
    \item отправка данных;
    \item ожидание повторной передачи данных;
    \item повторная передача данных.
\end{itemize}

\nomenclaturex{GoCB}{GOOSE Control Block}{блок управления GOOSE}

\begin{figure}[ht]
    \centering
    \includegraphics[width=.5\linewidth]{fsmSenderGOOSE}
    \caption{Конечный автомат отправителя для сервиса GOOSE}
    \label{pic::domain::fsm_sender_goose}
\end{figure}

Переходы состояний конечного автомата на рисунке~\ref{pic::domain::fsm_sender_goose}
происходят по следующему алгоритму:
\begin{enumerate_num}
    \item Для GoEna установлено значение TRUE.
    \item Отправитель совершает GOOSE-запрос. Таймер повторной передачи запускается
    на основе timeAllowedToLive, который установлен отправителем. Значение параметра
    SqNum устанавливается равным нулю. Предполагается, что таймер повторной передачи
    должен быть в два раза меньше параметра timeAllowedToLive.
    \item Таймер истечения повторной передачи указывает время для повторной
    передачи. SqNum увеличивается, пропуская 0 на переполнении.
    \item Повторно выполняется GOOSE-запрос. Используется следующий интервал
    повторной передачи. Запускается таймер повторной передачи. Метод выбора
    интервалов повторной передачи, как и максимальное время, допустимое между
    повторными передачами, являются локальными вопросами и устанавливаются
    на усмотрение отправителя. Допустимое время между повторными передачами должно
    составлять менее 60 секунд.
    \item При обнаружении изменения значения одного из элементов DataSet,
    StNum увеличивается, а SqNum устанавливается равным нулю.
    \item Все сообщения GOOSE и повторные передачи останавливаются, когда для GoEna
    установится значение FALSE.
\end{enumerate_num}

Получатель должен создать конечный автомат, приведенный
на рисунке~\ref{pic::domain::fsm_receiver_goose}, состоящий из трех состояний:
\begin{itemize}
    \item данные не существуют;
    \item значение данных действительно;
    \item вероятное значение данных.
\end{itemize}

\begin{figure}[ht]
    \centering
    \includegraphics[width=.4\linewidth]{fsmReceiverGOOSE}
    \caption{Конечный автомат получателя для сервиса GOOSE}
    \label{pic::domain::fsm_receiver_goose}
\end{figure}

\fixTableSectionSpace

\begin{table}[ht]
    \caption{Параметры структуры DstAddress}
    \label{table:domain:dst_addr_types}
    \begin{tabular}{| >{\raggedright}m{0.137\textwidth}
                    | >{\raggedright\arraybackslash}m{0.81\textwidth}|}
        \hline
        \centering Поле & \centering\arraybackslash Описание \\

        \hline
        Addr &
        Длина составляет 6 октетов и содержит значение MAC-адреса, на который
        должно быть отправлено сообщение GOOSE. Адрес должен быть адресом Ethernet,
        для которого бит многоадресной рассылки установлен в значение TRUE. \\

        \hline
        PRIORITY & Приоритет пакета. Диапазон значений ограничен от 0 до 7. \\

        \hline
        VID & Идентификатор VLAN. Диапазон значений ограничен от 0 до 4095. \\

        \hline
        APPID & Идентификатор приложения. Алгоритм выбора значения описан
        в приложении C стандарта \iecStdRef81. \\

        \hline
    \end{tabular}
\end{table}

\nomenclaturex{MAC}{Media Access Control}{контроль за доступом к среде передачи данных}

Переходы состояний конечного автомата на
рисунке~\ref{pic::domain::fsm_receiver_goose} происходят по следующему алгоритму:
\begin{enumerate_num}
    \item Получатель принимает корректное GOOSE-сообщение и запускает таймер
    истечения срока действия timeAllowedToLive.
    \item Истекает срок действия таймера timeAllowedToLive.
    \item Получатель принимает корректное GOOSE-сообщение с новыми или
    повторяющимися данными.
\end{enumerate_num}

\subsection{Сопоставление полей протоколов GOOSE и MMS}

MMS вместе с GOOSE являются основными протоколами передачи данных стандарта
\iecStd. Он использует технологию передачи клиент-сервер, не является
коммуникационным протоколом, а только определяет сообщения, которые должны
передаваться по сети. В качестве коммуникационного протокола в MMS используется
стек TCP/IP.

Типы GOOSE должны быть сопоставлены с типами MMS~\cite{IEC61850_7_2}.
В таблице~\ref{table:domain:goose_mms_equality} приведено описание параметров,
которое используется при передаче GOOSE средствами MMS.

\begin{table}[ht]
    \caption{Параметры GOOSE при использовании MMS}
    \label{table:domain:goose_mms_equality}
    \begin{tabular}{| >{\raggedright}m{0.137\textwidth}
                    | >{\raggedright\arraybackslash}m{0.81\textwidth}|}
        \hline
        \centering Поле & \centering\arraybackslash Описание \\

        \hline
        GoEna & Должно соответствовать стандарту \iecStdRef72. \\

        \hline
        GoID & Значение по умолчанию этого атрибута должно быть ссылкой на GoCB. \\

        \hline
        DatSet &
        Должно иметь тип данных ObjectReference. Значение должно быть ограничено
        набором существующих списков NamedVariableList MMS. Значения, указывающие
        на несуществующий NamedVariableList, должны считаться ошибочными. \\

        \hline
        ConfRev &
        Беззнаковое целочисленное значение в диапазоне от 0 до $ 4294967295 $. \\

        \hline
        NdsCom &
        Этот компонент MMS представляет атрибут NdsCom стандарта \iecStdRef72. \\

        \hline
        DstAddress &
        Структурированный тип MMS, компоненты которого определены в соответствии
        с таблицей~\ref{table:domain:dst_addr_types}. \\

        \hline
        MinTime &
        Задержка отправки при изменении данных между первой отправкой и
        первым повторением. Единица измерения -- миллисекунды. \\

        \hline
        MaxTime &
        Время контроля источника в миллисекундах. В течение этого времени клиент
        должен обнаружить ошибочное сообщение от источника, если такое было. \\

        \hline
    \end{tabular}
\end{table}

\fixTableSectionSpace

% \subsection{Существующие аналоги}

% TODO: implement
