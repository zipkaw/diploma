\sectionCenteredToc{ВВЕДЕНИЕ}
\label{sec:intro}

В настоящее время во многих приборах и устройствах используются инерциальные датчики, 
позволяющие получать полезные данные о движении объекта.
Благодаря своей способности измерять движение без необходимости во внешних ориентирах, 
инерциальные датчики находят применение в самых разных областях. Они лежат в основе навигационных 
систем для беспилотных летательных аппаратов (БПЛА) и автономных мобильных средств, обеспечивая 
точное определение местоположения и ориентации. В робототехнике эти датчики помогают роботам 
ориентироваться в пространстве, координировать движения и выполнять сложные манипуляции.
В данных приложениях датчики играют решающую роль в их работе.

Независимо от области применения, качество данных, получаемых от инерциальных датчиков, имеет решающее значение.
К сожалению, инерциальные датчики склонны к накоплению ошибок во времени. Это обусловлено несколькими факторами, 
такими как шум измерений и дрейф нуля: датчики всегда подвержены влиянию шума и смещению данных во времени даже если датчик не двигается.
Поэтому разработка методов и алгоритмов, позволяющих минимизировать влияние этих факторов и повысить точность и надежность инерциальных датчиков,
является важной задачей, решение которой будет способствовать дальнейшему развитию и совершенствованию широкого спектра современных технологий.

Целью данного дипломного проекта является разработка модуля приема и обработки данных от инерциальных датчиков,
таких как магнетометр, акселерометр и гироскоп, для тепловизионного наблюдательного прибора <<ДОZОР>>~.
Прием данных будет обеспечен драйверами работающими на микроконтроллере.
Обработка данных будет состоять из фильтрации и калибровки для преобразования искаженных данных в достоверные. 
Фильтрация состоит из двух частей, первый из них это фильтр Мэджвика, который позволяет объединить в себе данные с 
акселерометра и гироскопа и выровнять их. Вторая часть это БИХ-фильтр, в приложении он используется для обработки данных только с гироскопа
и позволяет определить повороты в пространстве избегая биения и резкие движения. 
Калибровка реализована отдельно под каждый датчик и состоит из двух частей: вычисление калибровочных параметров и применение параметров к данным.
Вычисление калибровочных параметров использует математически обоснованные методы. 

Подход к разработке модульной архитектуре позволяет создавать платформно-независимые алгоритмы, который может работать как на разных 
микроконтроллерах, так и с различными инерционными датчиками. Использование алгоритмов рекомендуется к использованию вместе с 
бюджетными датчиками склонными к искажениям и чувствительным к
внешней среде. 

Недостатками программного средства являются большое количество операций связанных с плавающей точкой на этапе вычисления калибровочных параметров,
а также на этапе калибровки данных. Также можно выделить, то что отображение ориентации и азимута требовательно к памяти программы.   

В соответствии с поставленной целью были определены следующие задачи:
\begin{enumerate_num}
    \item Исследование и анализ методов фильтрации и калибровки датчиков.
    \item Проектирование модуля приема и обработки данных.
    \item Разработка блоков приема и обработки датчиков.
    \item Тестирование работоспособности модуля.
    \item Расчет экономических показателей дипломного проекта.
    \item Написание руководства пользователя.
\end{enumerate_num}
