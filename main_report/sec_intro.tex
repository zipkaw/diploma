\sectionCenteredToc{ВВЕДЕНИЕ}
\label{sec:intro}

В ответственной профессии инженера-системотехника важно всегда пользоваться
качественным оборудованием, устойчивыми к воздействию окружающей среды компонентами,
защищенными системами, надежными протоколами связи. Особенно это необходимо для тех,
кто работает со взрывоопасными веществами, радиацией, сверхточными вычислениями,
ведь ошибки в таких вещах влекут катастрофы и ужасающие последствия. Понимая серьезность всего происходящего, были созданы институт инженеров электротехники и электроники, международный союз электросвязи, международная электротехническая комиссия, международная организация по стандартизации и другие, которые стали разрабатывать стандарты по радиоэлектронике, электротехнике, аппаратному и программному обеспечению вычислительных систем и сетей
с целью предотвращения катастроф. Стандартизация позволила достичь единства расчетов, формул, структур, единиц измерения посредством установления положений для всеобщего и многократного применения в отношении реально существующих или потенциальных задач.

Целью данного дипломного проекта является разработка модуля приема и обработки GOOSE-пакетов. Протокол GOOSE является одним из трех основных протоколов стандарта \iecStdRef81. GOOSE-протокол предназначен для обмена информацией между устройствами РЗА в цифровом виде, а их интеграция в систему производится
с помощью протокола MMS, сертифицированного \iso, который единственный имеет доказанный
практический результат легкой работы со сложным присваиванием имен и моделей
сервиса по стандарту \iecStd. Основными характеристиками GOOSE протокола являются надежность, чувствительность, селективность, быстродействие. Данная модель обеспечивает механизм передачи событий (например, команды и предупреждения) и используется для отключения, запуска, записи аварийных событий. Важной особенностью протокола является гарантированная доставка сообщений. Скорость передачи данных выше, чем у других протоколов передачи данных схожего назначения, к примеру, Modbus. Системы, построенные с использованием протокола GOOSE, проще обслуживать ввиду уменьшения количества кабельных соединений, что положительно сказывается на надежности системы в целом. Из этого следует, что архитектура сети сильно упрощается, в результате разработчики тратят минимальное количество времени на понимание архитектуры конкретного объекта и, как следствие, значительно снижается стоимость проектирования и интеграции.

\nomenclatureRus{РЗА}{релейная защита и автоматика}
\nomenclaturex{IEC}{International Electrotechnical Commission}{международная электротехническая комиссия, МЭК}
\nomenclaturex{GOOSE}{Generic Object Oriented Substation Event}{обобщенное объектно-ориентированное событие подстанции}

Специалисты в области радиоэлектроники считают протокол GOOSE технологией будущего.
Благодаря своим характеристикам он все чаще используется на новых энергообъектах,
следовательно, есть заказы и задачи на реализацию и взаимодействие с протоколом к различным IT-компаниям и необходимы специалисты,
способные выполнять данные задачи. Протокол получил широкое распространение за рубежом, активно развивается в странах постсоветского пространства.
Специализированная аутсорсинговая IТ-организация ООО <<АКСОНИМ>>, занимающаяся разработкой программных и аппаратных средств по индивидуальным заказам,
разработкой встроенных систем, электроники, цифровых устройств, в которой работает автор, неоднократно получала заказы на реализацию оборудования
для новейших энергообъектов с возможностью контроля работы с использованием GOOSE-протокола. Продукты,
разработанные командами специалистов компании, находят широкое применение в современном мире.

Недостатками протокола являются повышенная сложность и высокая стоимость микропроцессорного оборудования РЗА, которые объясняются его универсальностью.

В соответствии с поставленной целью были определены следующие задачи:
\begin{enumerate_num}
    \item Исследование и анализ протокола GOOSE.
    \item Проектирование модуля приема и обработки GOOSE-пакетов.
    \item Разработка блоков приема и обработки GOOSE-пакетов.
    \item Тестирование работоспособности модуля.
    \item Расчет экономических показателей дипломного проекта.
    \item Написание руководства пользователя.
\end{enumerate_num}
