\section{РУКОВОДСТВО ПОЛЬЗОВАТЕЛЯ}
\label{sec:manual}

В данном разделе описана информация, которая необходима пользователям
разрабатываемой системы обработки инерциальных датчиков для ее корректного использования.

Разрабатываемый продукт для компании OOO <<АКСОНИМ>> является библиотекой, которая включена
в итоговый продукт и не предполагает возможности быть самодостаточным программным обеспечением.
Данный раздел описывает все необходимые этапы для успешной сборки программного обеспечения, который может быть включен
в итоговый продукт.

\subsection{Общее описание особенностей}

Разрабатываемая в контексте данного дипломного проекта
часть системы поставляется в виде исходных кодов и способна работать
на платформах без операционной системы, также предполагается, что данный продукт может работать
на операционной системе реального времени, такой как FreeRTOS, так и
на обычной операционной системе, такой как Linux. Однако данный продукт в рамках дипломного проектирования был
протестирован только на софт-процессоре Micriblaze без использования операционной системы.

Для того чтобы код был переносимым на другие платформы и операционные системы, для работоспособности модулей
и возможности обращаться с драйверами платформы, данным программным продуктом используется библиотека OSAL, которая
является уровнем абстракции между операционной системой и приложением, и позволяет использовать функции драйверов, которые
могут отличаться от платформы или операционной системы.

Также для того чтобы запустить данное программное обеспечение на целевом устройстве, на нем должна находится аппаратная и программная поддержка таких интерфейсов
как SPI и IIC. А также должны быть настроены система видеотракт и буферы для отображения информации. Данные драйвера в исполняемом коде целевой платформы должны быть
обернуты в функции библиотеки OSAL, для того чтобы быть вызываемыми из библиотек более высокого уровня, например, такого как реализуемый дипломный проект.

\nomenclaturex{OSAL}{Operating System Abstraction Layer}{уровень абстракции операционной системы}

\subsection{Требования к аппаратному и программному обеспечению}

Для сборки и дальнейшего использования данного дипломного проекта необходимо
следующее программное обеспечение:
\begin{itemize}
    \item операционная система Linux, на котором будет производится сборка пакетов;
    \item программное обеспечение Vivado и Vitis версии 2020.1 от компании Xilinx, со всеми установленными пакетами.
    \item печатная плата выпущенная компанией OOO <<АКСОНИМ>> или отладочная плата  на базе FPGA Artix-7 от компании Xilinx с несколькими интерфейсами SPI и IIC, 
    сенсорами MEMSIC и STM, а так же с флэш памятьюж;
    \item кроссплатформенная система сборки CMake версии 3.20.0 или новее
    для автоматизации сборки целевого проекта;
    \item среда исполнения языка программирования Ruby версии не ниже 3.1.0
    для генерации вспомогательных файлов тестирования пакета;
    \item основная система сборки Make версии не ниже 4.0 для выполнения всех команд,
    сгенерированных CMake;
    \item компилятор для языка C с поддержкой стандарта C99 для создания объектных
    файлов из исходных кодов;
\end{itemize}

% TODO: Добавить информацию о сборке на других операционных системах 

\subsection{Настройка рабочего окружения}

Основной платформой, которая заявлена в требованиях,
является операционная система семейства Linux.
В качестве примера, на котором будет показана установка всех необходимых
зависимостей, была взята Ubuntu 22.04 LTS amd64 как самая свежая версия одного из самых популярных дистрибутивов на момент написания текущего руководства.

\nomenclaturex{LTS}{Long-Term Support}{увеличенный срок поддержки}

Порядок установки выбранной операционной системы описан в
источнике~\cite{ubuntu_how_to_install}. Несмотря на то, что там расписан порядок
установки для более старой версии данного дистрибутива, выбранный вариант
полностью соответствует новой версии.
Скачать необходимую версию дистрибутива можно с официального
сайта~\cite{ubuntu_download_site}.

Для установки Vitis и Vivado.... % TOOD

\subsection{Сборка исходных кодов}

Для сборки разработанной библиотеки необходимо скопировать ее с диска, поставляемого
с дипломным проектом, на компьютер. Сделать это можно как средствами
графического интерфейса, так и с помощью утилит командной строки.

\begin{table}[ht]
    \caption{Возможные типы сборки Conan и CMake}
    \label{table:manual:buildTypes}
    \begin{tabular}{| >{\raggedright}m{0.27\textwidth}
                    | >{\raggedright\arraybackslash}m{0.677\textwidth}|}
        \hline
        \centering Название & \centering\arraybackslash Описание \\

        \hline
        Debug &
        Отладочная сборка
        \\

        \hline
        Release &
        Сборка для выпуска
        \\

        \hline
        RelWithDebInfo &
        Сборка для выпуска с отладочной информацией
        \\

        \hline
        MinSizeRel &
        Сборка для выпуска с минимизацией размера
        \\

        \hline
    \end{tabular}
\end{table}

\fixTableSectionSpace

Описать последовательность сборки проекта

Описать Запуск тестов и продемонстрировать результаты
