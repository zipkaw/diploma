\section{РУКОВОДСТВО ПОЛЬЗОВАТЕЛЯ}
\label{sec:manual}

В данном разделе описана информация, которая необходима пользователям
разрабатываемой системы обработки инерциальных датчиков для ее корректного использования.

Разрабатываемый продукт для компании OOO <<АКСОНИМ>> является библиотекой, которая включена
в итоговый продукт и не предполагает возможности быть самодостаточным программным обеспечением.
Данный раздел описывает все необходимые этапы для успешной сборки программного обеспечения, который может быть включен
в итоговый продукт.

\subsection{Общее описание особенностей}

Разрабатываемая в контексте данного дипломного проекта
часть системы поставляется в виде исходных кодов и способна работать
на платформах без операционной системы, также предполагается, что данный продукт может работать
на операционной системе реального времени, такой как FreeRTOS, так и
на обычной операционной системе, такой как Linux. Однако данный продукт в рамках дипломного проектирования был
протестирован только на софт-процессоре Micriblaze без использования операционной системы.

Для того чтобы код был переносимым на другие платформы и операционные системы, для работоспособности модулей
и возможности обращаться с драйверами платформы, данным программным продуктом используется библиотека OSAL, которая
является уровнем абстракции между операционной системой и приложением, и позволяет использовать функции драйверов, которые
могут отличаться от платформы или операционной системы.

Также для того чтобы запустить данное программное обеспечение на целевом устройстве, на нем должна находится аппаратная и программная поддержка таких интерфейсов
как SPI и IIC. А также должны быть настроены система видеотракт и буферы для отображения информации. Данные драйвера в исполняемом коде целевой платформы должны быть
обернуты в функции библиотеки OSAL, для того чтобы быть вызываемыми из библиотек более высокого уровня, например, такого как реализуемый дипломный проект.

Также возможно два способа запуска приграммы на целевом устройстве -- это используя JTAG либо используя загрузчик. Первый и второй способ предполагает наличие JTAG
разъема на печатной плате, а также наличие программатора. Первый способ годится для отладки программного обеспечения, а второй предполагает, что программное обеспечение
будет загружаться на старте.

\nomenclaturex{OSAL}{Operating System Abstraction Layer}{уровень абстракции операционной системы}

\subsection{Требования к аппаратному и программному обеспечению}

Для сборки и дальнейшего использования данного дипломного проекта необходимо
следующее программное обеспечение:
\begin{itemize}
    \item операционная система Linux, на котором будет производится сборка пакетов и прошивка устройства;
    \item программное обеспечение Vivado и Vitis версии 2020.1 от компании Xilinx, со всеми установленными пакетами.
    \item печатная плата выпущенная компанией OOO <<АКСОНИМ>>;
    \item кроссплатформенная система сборки CMake версии 3.20.0 или новее
    для автоматизации сборки целевого проекта;
    \item среда исполнения языка программирования Ruby версии не ниже 3.1.0
    для генерации вспомогательных файлов тестирования пакета;
    \item основная система сборки Ninja версии не ниже v1.12.1 для выполнения всех команд,
    сгенерированных CMake;
    \item компилятор для языка C с поддержкой стандарта C99 для создания объектных
    файлов из исходных кодов;
\end{itemize}

% TODO: Добавить информацию о сборке на других операционных системах 

\subsection{Настройка рабочего окружения}

Основной платформой, которая заявлена в требованиях,
является операционная система семейства Linux.
В качестве примера, на котором будет показана установка всех необходимых
зависимостей, была взята Ubuntu 22.04 LTS amd64 как самая свежая версия одного из самых популярных дистрибутивов на момент написания текущего руководства.

\nomenclaturex{LTS}{Long-Term Support}{увеличенный срок поддержки}

Порядок установки выбранной операционной системы описан в
источнике~\cite{ubuntu_how_to_install}. Несмотря на то, что там расписан порядок
установки для более старой версии данного дистрибутива, выбранный вариант
полностью соответствует новой версии.
Скачать необходимую версию дистрибутива можно с официального
сайта~\cite{ubuntu_download_site}.

Для установки Vitis и Vivado, что содержат необходимые инструменты для сборки пакетов, а также отладки и прошивки устройства, можно
обратиться за подробной инструкцией в источнике~\cite{install_vitis, install_vivado}, в процессе установки необходимо будет установить
все необходимые пакеты. Для установки CMake, Ruby на операционной системе Linux можно воспользоваться командой:
\begin{lstlisting}[basicstyle=\ttfamily\small]
    $ sudo apt update
    $ sudo apt install -y cmake ruby
\end{lstlisting}

Для установки Ninja необходимо его собрать из исходного кода, для этого:
\begin{lstlisting}[basicstyle=\ttfamily\small]
    $ git clone git://github.com/ninja-build/ninja.git && cd ninja
    $ git checkout release
    $ cmake -Bbuild-cmake
    $ cmake --build build-cmake
\end{lstlisting}

Чтобы проверить, что все установлено верно можно запустить тесты: 
\begin{lstlisting}[basicstyle=\ttfamily\small]
    $ ./build-cmake/ninja_test
\end{lstlisting}

\subsection{Сборка исходных кодов}

Для сборки разработанной библиотеки необходимо скопировать ее с диска, поставляемого
с дипломным проектом, на компьютер. Сделать это можно как средствами
графического интерфейса, так и с помощью утилит командной строки.

Для того чтобы собрать программное обеспечение и прошить целевое устройство необходимо создать проект в Vitis который будет управлять этапом прошивки устройства. Для
этого необходимо:

\begin{enumerate}
    \item Рекомендуется создать новое рабочее пространство в Vitis, чтобы избежать конфликта имен. 
    Для этого в панели инструментов выбрать File, далее выбрать Switch Workspace, далее Other..., 
    далее Browse, далее указать путь к пустой папке, где будет храниться новое рабочее пространство, нажать на кнопку Launch.
    \item Создать проект под нужную платформу в Vitis. Для этого в панели инструментов выбрать File, далее New, далее Platform Project, далее в 
    поле Platform project name ввести microblaze\_soc, далее Next, далее Browse и указать путь к нужному .xsa файлу, 
    они располагаются в папке board, в конце нажать на кнопку Finish.
    \item Импортировать все проекты в среду Vitis. Для этого в панели инструментов выбрать File, далее Import, далее Eclipse workspace or zip file, далее Next, далее
    напротив Select root directory выбрать Browse и указать путь к  папке с проектом, далее в поле Projects выбрать все проекты, а в Options 
    и Working sets снять все галочки и нажать на кнопку Finish.
\end{enumerate}

Для сборки проекта при помощи CMake необходимо ввести в терминал, находясь в папке с проектом следующие команды:

\begin{lstlisting}[basicstyle=\ttfamily\small]
    $ mkdir build && cd build
    $ cmake -D CMAKE_TOOLCHAIN_FILE="../cmake/mb_v11.cmake" -DBUILD_TYPE=Release -G Ninja .. && cmake --build .
\end{lstlisting}

Чтобы загрузить исполняемый файл в режиме отладки можно воспользоваться встроенной в Vitis консолью XCST и ввести следующие команды:

\begin{lstlisting}[basicstyle=\ttfamily\small]
    connect
    fpga -f < full path to bit file>
    targets 3
    dow <full path to elf file>
    con
\end{lstlisting}

В качестве <bit file full path> необходимо указать полный путь полный путь до файла microblaze\_soc.bit, который находится в папке microblaze\_soc.
В качестве <full path to elf file>> необходимо указать полный путь полный путь до файла MicroblazeNovo.elf, который находитсяв папке build/microblaze\_workspace.

Чтобы загрузить прошивку во флэш память необходимо сначала загрузить загрузчик в начальный адрес памяти, для этого в консоли XCST нужно пропсать:

\begin{lstlisting}[basicstyle=\ttfamily\small]
    program_flash -f <full path to mcs file> -offset 0x00000000 -flash_type s25fl128sxxxxxx0-spi-x1_x2_x4 -blank_check -verify -cable type xilinx_tcf url TCP:127.0.0.1:3121
\end{lstlisting}

В качестве <full path to mcs file> необходимо указать полный путь полный путь до файла novo\_loader.mcs, который находитсяв папке build.

Далее после успешной операции прошивки флэш памяти загрузчиком, необходимо прошить флэш память исполняемым файлом, делается это также через консоль XCST: 

\begin{lstlisting}[basicstyle=\ttfamily\small]
    program_flash -f <full path to elf file> -offset 0x00800000 -flash_type s25fl128sxxxxxx0-spi-x1_x2_x4 -blank_check -verify -cable type xilinx_tcf url TCP:127.0.0.1:3121
\end{lstlisting}

Далее можно перезагрузить целевое устройство и прошивка автоматически будет загружена. 

\subsubsection{Сборка только библиотеки модуля}

Для создания библиотеки модуля обработки инерциальных датчиков, потребуется запустить следующий скрипт
(предполагается, что пользователь находится в папке с исходным кодом библиотеки):

\begin{lstlisting}[basicstyle=\ttfamily\small]
    mkdir build && cd build
    cmake -G Ninja ..
    cmake --build .
\end{lstlisting}

Также автоматически будут собраны тесты, для их запуска потребуется выполнить следующую команду:

\begin{lstlisting}[basicstyle=\ttfamily\small]
    cd build/test
    ctest --rerun-failed --output-on-failure
\end{lstlisting}
