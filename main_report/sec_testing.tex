\section{ПРОГРАММА И МЕТОДИКА ИСПЫТАНИЙ}
\label{sec:testing}

В данном разделе рассмотрено тестирование разработанного программного обеспечения.
Его необходимо проводить во время разработки продукта, а также перед выпуском
с целью выявления ошибок и их устранения, для проверки соответствия продукта
заявленным требованиям, оценки качества работы разработчиков. Тестирование
обеспечивает безопасность кода при командной работе, помогает в создании наилучшей
архитектуры, улучшает качество кода, упрощает выявление ошибок и экономит денежные
средства организации.

В ходе работы проводилось тестирование ключевых модулей программного продукта. 
Большинство тестов проверяют калибровку и обработку данных на корректность работы
и качество предоставляемых данных. Тесты проверяют работу отдельных модулей, и то как
другие модули влияют на проверяемый модуль. В данном разделе внимание будет отведено
тестированию модулей калибровки и расчета ориентации и азимута, так как в 
этих модулях заключается важная работа продукта.

Для проведения модульного тестирования используется библиотека Unity. Она предоставляет
большое количество способов проверки корректности значений и написана с использованием
стандарта C99. 

Также для того чтобы автоматизировать тестирование был разработан скрипт
СMake, который находит в корне проекта названия исходных файлов с суффиксом test.c, добавляет их к
файлам теста проекта и собирает исполняемые файлы для каждого тестируемого модуля.

Предоставление возможности использования функций других модулей системы без реализации
предоставляет библиотека CMock от разработчиков Unity. Она написана на языках программирования
C и Ruby и способна генерировать реализации функций по их прототипам. Библиотека
производит создание необходимых реализаций в соответствии со
стандартом C99, а также имеет гибкие
возможности по установлению соответствия пользовательских типов с типами стандартной библиотеки
языка C.

\subsection{Тест модуля \moduleCalib}

В данном разделе приведено тестирование модулей калибровки. Для каждого модуля отличается
последовательность калибровки, поэтому тесты рассматриваются по-отдельности. 
Однако для тестирования каждого модуля был выбран единый подход:
был собран набор данных каждого модуля, для каждого набора данных был рассчитан эталонный откалиброванный набор.
Если в тесте выходные данные модуля отклонялись на определенное значение, то тест считается не пройденным.
Помимо тестирования математических расчетов, в тестах проводится проверка работы конфигурации модулей,
а также проверка взаимодействия с другими модулями, сильно влияющих на работу самого модуля.

\subsubsection{Тесты модуля калибровки магнитометра}

Тест \lstinline|test_calibration_open_badChecksum|, производит создание дескриптора
и инициализирует модуль, в параметры функции которого передается обратный вызов, проверяющий код возврата после чтения флэш памяти параметров, а
также проверяет данные которые возвращает поток.
На данном этапе проверяются выходные параметры, такие как полученный дескриптор и код возврата. 
Далее запускается главный поток модуля.
Замаскированная, при помощи библиотеки CMock, функция чтения из флэш памяти возвращает данные с неверным CRC кодом. После этого проверяются
данные о калибровочных параметрах записанные в дескриптор модуля. Целью данного теста является проверка поведения в случае вычитки неверных параметров,
а так же как эти параметры используются дальше.

Тест \lstinline|test_calibration_open_goodChecksum|, производит создание дескриптора и инициализирует модуль как и в предыдущем случае, отличается
только функция обратного вызова, которая проверяет правильность полученного CRC кода.
В данном тесте используется замаскированная функция, которая возвращает после чтения параметров верный СКС код. Далее вызывается поток, в котором происходит
выполнения этапа вычитки данных из флэш памяти, и дальнейшее применение вычитанных параметров на наборе данных. Целью данного теста является проверка
правильности полученных данных и то что вычитанные параметры применяются на наборе данных правильно.

Тест \lstinline|test_calibration_open_flashAddress|, производит начальные этапы настройки модуля как в тестах выше. Библиотека CMock, позволяет
проверять параметры, которые передаются в функцию чтения параметров. Адрес проверятся тот, который определен в платформе специально для чтения 
и записи параметров магнетометра. Цель данного теста защитить код, от изменения оригинального адреса для записи параметров в коде.

В следующих тестах, создание дескриптора и инициализация была произведена заранее. В начале каждого теста модуль проинициализирован стандартными параметрами.

Тест \lstinline|test_calibration_applyDefParamMultipleTimes| проверяет, что при стандартной калибровочной матрице, параметры применяются правильно. 
В данном тесте на вход подается необработанный набор данных, поток обрабатывает их, и на выходе из теста ожидается получение таких же данных, что были
поданы на вход. Целью данного теста является проверка, что поток применяет калибровочные данные при стандартных параметрах, так как и предполагается.

Тест \lstinline|test_calibration_recalibrate| проверяет выполнение команды перекалибровки модуля. Модуль обрабатывает получаемые на вход данные,
производит операцию вычисления параметров и записывает данные во флэш память (вызовы для записи замаскированы). При записи тест проверяет, что данные
вычислены так как ожидается и, что CRC-код правильный.

Тест \lstinline|test_calibration_applyParamMultipleTimes| проверяет, что при вычисленных калибровочных параметрах, модуль возвращает верные данные. 
В данном тесте на вход подается набор данных для калибровки и эти данные обрабатывает тест описанный выше, далее на вход подаются неоткалиброванный набор данных
и на выходе ожидается получение верных откалиброванных параметров. 
Целью данного теста является проверка, что модуль правильно применяет процесс калибровки и возвращается к состоянию обработки входных данны, что верно калибрует
данные, при новых полученных параметрах.

Для данного модуля тесты описанные выше являются наиболее важными в ходе работы, так как вся основная работа заключается именно в последовательностях, что описано выше.
Другие тесты, на которых не было акцентировано внимание, описывают правильность переходов состояний модуля при различных внешних воздействиях, такие как получение ошибки
на каждом этапе калибровки, либо каким образом себя поведет модуль при отмене калибровки, либо как поведет себя модуль при неправильном внешнем воздействии. Все эти 
тесты позволяют гарантировать надежность работы модуля при любых условиях.

\subsubsection{Тесты модуля калибровки акселерометра}

В ходе тестирования проверяются те же сценарии инициализации и создания дескриптора, что и в тестах модуля калибровки магнетометра. Однако для данного модуля следует
рассмотреть свои особые сценарии использования так как они отличаются от тех, что представлены в модуле калибровки магнетометра.

Тест \lstinline|test_bl_calibrAcc_applyParam| проверят, что данные полученные на входе были правильно откалиброваны. 
Также в данном тесте обрабатываются случаи получения ошибки от драйвера магнетометра и проверяется состояние модуля. Цель данного теста проверить
корректность работы алгоритма калибровки данных.

Тест \lstinline|test_bl_calibrAcc_recalibSetPosWhenCollectAll|, проверяет, поведение модуля, когда происходит последовательность калибровки акселерометра.
В данном сценарии проверятся поведение модуля, когда установлено 2 позиции из 6, и в качестве 3 позиции приходят некорректные параметры. А также в случае успешного
прохождения данного шага проверяется установка позиции, когда установлено все 6 позиций из 6. В данном случае ожидается, что модуль вернет пользователю ошибку с неверным
переданным параметром, исключая возможности повторной перекалибровки. Цель данного тестового сценария выявить, что последовательность позиций прибора задана правильно, и что, когда
модуль находится в состоянии калибровки, верно обрабатывается порядок позиций.

Тест \lstinline|test_bl_calibrAcc_recalibCancelWhenCollectData|, проверяет поведение модуля при отмене калибровки. Ожидается, что прием данных завершится,
модуль перейдет в обычный режим работы и параметры будут прежние, что и до процесса перекалибровки. Также ожидается сброс состояний и счетчика данных. Цель данного
теста заключается в проверке прерывания процесса калибровки и сброса всех параметров, которые могут повлиять на обычный режим работы модуля либо на расчет новых калибровочных параметров.

Выше были рассмотрены ключевые тесты для данного модуля, данные сценарии в готовом продукте являются наиболее используемыми. Однако в данном модуле были описаны и другие
сценарии, которые необходимо упомянуть. Были проверены сценарии обработки ошибок внешних модулей, таких как флэш память и датчик. Было проверено поведение модуля при
ожидании ответа  от внешних модулей, с ожиданием того, что модуль не должен ничего возвращать и должен находится в одном состоянии на протяжении всего времени.
Также были проведены проверки работы приходящих команд, и реакция модуля на эти команды. Ожидается верное установленно состояние и верный переход в следующее состояние.

\subsubsection{Тесты модуля калибровки гироскопа}

Тестирование данного модуля включает те же сценарии что и в предыдущих случаях. Также проверяется инициализация и создание дескриптора модуля, проверяется
последовательность выполнения команд, правильность вычисления калибровочных параметров и правильность калибровки набора данных. Так как данный модуль обладает
простым механизмом калибровки, нет особенного сценария для поиска ошибок.  

\subsection{Тесты модуля \moduleOrientationAzimuth}

Тесты данного модуля покрывают возможные сценарии использования модуля. Это проверки на получение данных, вычисление ориентации при определенных положениях.

Тест \lstinline|test_bl_devPos_notAllMeasInit|, проверяет ожидаемое состояние модуля в случае когда не получены все данные. В данном тесте проверяются все возможные 
комбинации того как данные могут быть получены: от случая, когда ни один из модулей внешних не передал данные в модуль \moduleOrientationAzimuth, до случая, когда только
один из внешних модулей не передал данные. В каждом из случаев ожидается одно единственное поведения, когда модуль ожидает получения новых данных одновременно.
Цель данного теста заключается в гарантии того, что модуль использует данные полученные в один момент времени, для корректного определения позиции и ориентации.

Тесты типа \lstinline|est_bl_devPos_staticDataHorizontal<angle>|, где \lstinline|<angle>| значение азимута в градусах, проверяют, что при определенном истинном наборе
входных данных, вычисленное значение азимута, равняется значению \lstinline|angle|. Цель теста является проверка, что вычислено верное значение азимута при различных положениях прибора.

Тесты на определения ориентации не предоставлены, так как вычислением ориентации прибора в Эйлеровской системе координат занимается сторонняя библиотека,
которая протестирована для данных случаев.

\subsection{Тесты модуля \moduleMoveDetect}

В данных тестовых сценариях используются наборы данных полученных с гироскопа мобильного телефона, который вращался с различной скоростью и в различных осях, а также покоился.
Также для достоверности наборов данных для тестирования они были проанализированы при помощи графической библиотеки matplotlib на соответствие того, что данный набор можно оценивать
как тот или иной сценарий.

Тест \lstinline|test_bl_devMoveDetector_notMove|, данный тест получает на вход нефильтрованные данные и возвращает на входе результат о движении прибора.
Данный сценарий предполагает, что принимаемый поток данных свидетельствует о том что прибор не движется. Цель теста является определение, что при небольших
биениях или резких движениях прибор будет определять состояние как покоящийся.

Тест \lstinline|test_bl_devMoveDetector_moveAfterNotMove|, данный тест получает на вход нефильтрованные данные и возвращает на входе результат о движении прибора.
Данный сценарий предполагает, что принимаемый поток данных свидетельствует о том что прибор движется после состояния покоя. Цель теста является определение, что
после состояния покоя, модуль определит то, что прибор начал движение и, что реакция фильтра на возникновение движения соответствует 100 миллисекундам, что приемлемо
для пользователя.

Тест \lstinline|test_bl_devMoveDetector_notMoveAfterMove|, проверяет таким же методом как и предыдущий тест, однако в данном
случае ожидается выходное состояние, что прибор покоится после движения.
